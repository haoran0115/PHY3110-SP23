%! TeX program    = xelatex
%! TEX-TS program = xelatex
\documentclass[twoside,11pt]{article}
\usepackage[left=1in, right=1in, top=1in, bottom=1in]{geometry}
\usepackage{amsmath}
\usepackage{amssymb}
\usepackage{amsfonts}
\usepackage{mathtools}
\usepackage{amsthm}
\usepackage{fancyhdr}
\usepackage{enumitem}
\usepackage{siunitx}
\usepackage{booktabs}
\usepackage[hidelinks]{hyperref}
\usepackage{sectsty}
\usepackage{mathrsfs} % mathscr
\usepackage{tikz}
\usepackage{pgfplots}
\usepackage{multicol}
\usepackage{listings}
% \usepackage{amsart}
\usepackage{fontspec}
\usepackage{soul}


% allow H option of figure
\usepackage{float}

% math font (libertine)
\usepackage{libertinus-otf}

% braket
\usepackage{braket}

% tikz library
\usetikzlibrary{decorations,calligraphy,calc,external}
\tikzexternalize
\tikzexternaldisable

% physics
% \usepackage{physics}

% define latin modern font environment
\newcommand{\lms}{\fontfamily{lmss}\selectfont} % Latin Modern Roman
% \newcommand{\lmss}{\fontfamily{lmss}\selectfont} % Latin Modern Sans
% \newcommand{\lmss}{\fontfamily{lmtt}\selectfont} % Latin Modern Mono

% % change mathcal shape
% \usepackage[mathcal]{eucal}


% define math operators
\newcommand{\FF}{\mathbb{F}}
\newcommand{\RR}{\mathbb{R}}
\newcommand{\NN}{\mathbb{N}}
\newcommand{\ZZ}{\mathbb{Z}}
\newcommand{\QQ}{\mathbb{Q}}
\newcommand{\XX}{\mathbb{Y}}
\newcommand{\CL}{\mathcal{L}}
% \renewcommand{\d}{\mathrm{d}}
\renewcommand*\d{\mathop{}\!\mathrm{d}}
\DeclareMathOperator*{\argmax}{arg\,max}
\DeclareMathOperator*{\argmin}{arg\,min}
\DeclareMathOperator{\im}{im}
\DeclareMathOperator{\id}{id}
\DeclareMathOperator{\erf}{erf}
\renewcommand{\mod}[1]{\ (\mathrm{mod}\ #1)}

% section font style
\sectionfont{\sffamily\Large}
\subsectionfont{\sffamily\normalsize}
\subsubsectionfont{\bf}

% line spreading and break
\hyphenpenalty=5000
\tolerance=20
\setlength{\parindent}{0em}
\setlength\parskip{0.5em}
\allowdisplaybreaks
\linespread{0.9}

% enumerate settings
% no break before enumerate
\setlist[enumerate]{itemsep=2pt,topsep=2pt}

% theorem
% latex theorem
% definition style
\theoremstyle{definition}
\newtheorem{theorem}{\lms Theorem}[subsection]
\newtheorem{axiom}{\lms Axiom}[section]
\newtheorem{definition}{\lms Definition}[section]
\newtheorem{example}{\lms Example}[section]
\newtheorem{question}{\lms Question}[section]
\newtheorem{exercise}{\lms Exercise}[section]
\newtheorem*{exercise*}{\lms Exercise}
\newtheorem{lemma}{\lms Lemma}[section]
\newtheorem{proposition}{\lms Proposition}[section]
\newtheorem{corollary}{\lms Corollary}[section]
\newtheorem*{theorem*}{\lms Theorem}
\newtheorem{problem}{\lms Problem}
% remark style
\theoremstyle{remark}
\newtheorem*{remark}{\lms Remark}
\newtheorem*{solution}{\lms Solution}
\newtheorem*{claim}{\lms Claim}


% paragraph indent
\setlength{\parindent}{0em}
\setlength\parskip{0.5em}

\newcommand\Code{PHY3110 SP23}
\newcommand\Ass{HW05}
\newcommand\name{Haoran Sun}
\newcommand\mail{haoransun@link.cuhk.edu.cn}

\title{{\lms \Code \ \Ass}}
\author{\lms \name \ (\href{mailto:\mail}{\mail})}
\date{\sffamily \today}

\makeatletter
% \let\Title\@title
\let\theauthor\@author
\let\thedate\@date

\fancypagestyle{plain}{%
    \fancyhf{}
    \lhead{\lms \Ass}
    \rhead{\lms \name}
    \rfoot{\lms\thepage}

    % # 页脚自定义
    \fancyfoot[L]{
        \begin{minipage}[c]{0.06\textwidth}
            \includegraphics[height=7.5mm]{logo2.png}
        \end{minipage}
    }
}
\fancypagestyle{title}{%
    \fancyhf{}
    \renewcommand{\headrulewidth}{0pt}
    % \lhead{\Title}
    % \rhead{\theauthor}
    \rfoot{\lms\thepage}

    % # 页脚自定义
    \fancyfoot[L]{
        \begin{minipage}[c]{0.06\textwidth}
            \includegraphics[height=7.5mm]{logo2.png}
        \end{minipage}
    }
}
\fancyfootoffset[L]{0.3cm}

% re-define title format
\makeatletter
\renewcommand{\maketitle}{\bgroup\setlength{\parindent}{0pt}
\begin{flushleft}
  \textbf{\Large\@title}

  \@author
\end{flushleft}\egroup
}
\makeatother

\pagestyle{plain}

% lstlisting settings
\lstset{
    basicstyle=\linespread{0.7}\footnotesize,
    breaklines=true,
    basewidth=0.5em
}


\begin{document}
\maketitle
\thispagestyle{title}
% \begin{multicols*}{2}

% \begin{remark}
%     $V_\epsilon(x)$ is used to denote a $\epsilon$-neighborhood
%     \begin{align*}
%         V_\epsilon(x) = B_\epsilon(x)\setminus\{x\}
%     \end{align*}
% \end{remark}

\begin{problem}
Find the orbits of a point mass moving in a central force field $F=-kr$,
where $k$ is a positive constant.
What if $k$ is a negative constant?
\end{problem}
\begin{solution}
Instead of using polar coordinate $(r, \theta)$, use the $(x, y)$ Cartesian coordinate system.
Then we have equations
\begin{align*}
    m\ddot x &= -kx,~ m\ddot y = -ky
\end{align*}
This leads to the solution
\begin{align*}
    x &= A\sin[\omega(t - t_1)],~ y = B\cos[\omega(t-t_2)]
\end{align*}
which is a ellipse (where $\omega = \sqrt{\frac{k}{m}}$).

Suppose $k$ negative, we have a differnt solution
\begin{align*}
    x &= A\sinh[\omega(t - t_1)],~ y = B\cosh[\omega(t-t_2)]
\end{align*}
which is a hyperbola (where $\omega = \sqrt{\frac{-k}{m}}$).

\end{solution}


\begin{problem}
A point mass $m$ moves in a central force field with $F=-\frac{\alpha}{r^2}$.
If its orbit is an ellipse with the semi-major axis $\alpha$, derive the following
relation between its velocity and $r$, $a$
\begin{align}
    v^2 &= 
    \alpha
    \left(
        \frac{2}{r} - \frac{1}{\alpha}
    \right)
\end{align}
\end{problem}
\begin{solution}
For ellipse orbits, $E<0$ and we have such relation
\begin{align*}
    a &= -\frac{k}{2E}
\end{align*}
In this case we have $k=\alpha m$, hence we have
\begin{align*}
    E &=  -\frac{\alpha m}{2a} = \frac{1}{2}m v^2 - \frac{\alpha m}{r}\Rightarrow
    v^2 = \alpha\left(\frac{2}{r} - \frac{1}{a}\right) 
\end{align*}

For parabola orbits, $E=0$ and hence
\begin{align*}
    E &= 0 = \frac{1}{2}m v^2 - \frac{\alpha m}{r}\Rightarrow
    v^2 = \frac{2\alpha}{r}
\end{align*}

For hyperbola orbits, $E>0$  and we have such relation
\begin{align*}
    E &=  \frac{\alpha m}{2a} = \frac{1}{2}m v^2 - \frac{\alpha m}{r}\Rightarrow
    v^2 = \alpha\left(\frac{2}{r} + \frac{1}{a}\right)
\end{align*}

\end{solution}

\begin{problem}
Consider the scattering produced by a repulsive force $F=\frac{k}{r^3}$,
show that the cross section takes the form
\begin{align}
    \sigma(\theta) &= 
    \frac{k\pi^2}{2E}
    \frac{(\pi - \theta)}{\theta^2(2\pi - \theta)^2\sin\theta}
\end{align}
\end{problem}
\begin{solution}
The potential energy takes the form $V=\frac{k}{r^2}$.
Using the relation
\begin{align*}
    E &= \frac{1}{2}m\dot r^2 + \frac{l^2}{2mr^2} + \frac{k}{2r^2}
\end{align*}
We know that $r$ takes its minimum $r_m$ when $\dot{r}=0$, therefore
\begin{align*}
    \left.
    \begin{aligned}
            l^2 &= 2mEs^2\\
            E &= \frac{l^2}{2mr_m^2} + \frac{k}{2r_m^2}
    \end{aligned}
    \right\}
    \Rightarrow
    r_m &= \left(S^2 + \frac{k}{2E}\right)^{1/2} = \frac{1}{u_m}
\end{align*}
Using the formula
\begin{align*}
    \psi 
    &= 
    \int_0^{u_m}
    \left( 1 - \frac{V}{E} - S^2 u^2 \right)^{-1/2}S\d u\\
    &=
    \int_0^{u_m}
    \left( 1 - \frac{ku^2}{2E} - S^2 u^2 \right)^{-1/2}S\d u\\
    &= 
    \int_0^{u_m}
    \left[ 1 - \left(\frac{k}{2E} + S^2\right)u^2 \right]^{-1/2}S\d u\\
    &= 
    S\left(\frac{k}{2E} + S^2\right)^{-1/2}
    \left.
        \arcsin\left(\frac{k}{2E} + S^2\right)^{1/2} u
    \right|_0^{u_m}\\
    &= \frac{\pi}{2}
    S\left(\frac{k}{2E} + S^2\right)^{-1/2}
\end{align*}
Hence we have $\theta$ equals to
\begin{align*}
    \theta 
    &= 
    \pi - 2\psi 
    = \pi\left[
        1 - 
    S\left(\frac{k}{2E} + S^2\right)^{-1/2}
    \right]
\end{align*}
Therefore
\begin{align*}
    S &= \left[
        \frac{k}{2E}\frac{(\pi - \theta)^2}{\theta(2\pi - \theta)}
    \right]^{1/2}
\end{align*}
Hence
\begin{align*}
    \frac{\d S}{\d\theta} &= 
    -S^{-1}\frac{k\pi^2}{2E}\frac{(\pi - \theta)}{\theta^2(2\pi - \theta)^2}
\end{align*}
Therefore we have the differential cross-section equals to
\begin{align*}
    \sigma(\theta) &= 
    \frac{S}{\sin\theta} \left|\frac{\d S}{\d \theta}\right|
    = \frac{k\pi^2}{2E}\frac{\pi - \theta}{\theta^2 (2\pi - \theta)^2\sin\theta}
\end{align*}

\end{solution}


\begin{problem}
Show that for an antisymmetric $3\times 3$ matrix $\mathbf{A}$,
the matrix $\mathbf{B}=(\mathbf 1 + \mathbf A)(\mathbf 1 - \mathbf A)^{-1}$
is orthogonal, where $\mathbf 1$ is the identity matrix.
\end{problem}
\begin{solution}
Note that
\begin{align*}
    \mathbf B^T 
    &= 
    [(\mathbf 1 - \mathbf A)^{-1}]^T(\mathbf 1 + \mathbf A)^T
    = 
    (\mathbf 1 + \mathbf A)^{-1} (\mathbf 1 - \mathbf A)\\
    \mathbf B^T \mathbf B 
    &= 
    (\mathbf 1 + \mathbf A)^{-1} (\mathbf 1 - \mathbf A)
    (\mathbf 1 + \mathbf A)(\mathbf 1 - \mathbf A)^{-1}
\end{align*}
Since $(\mathbf 1 + \mathbf A)(\mathbf 1 - \mathbf A) = (\mathbf 1 - \mathbf A)(\mathbf 1 + \mathbf A)$,
we have
\begin{align*}
    \mathbf B^T \mathbf B &=
    (\mathbf 1 + \mathbf A)^{-1} 
    (\mathbf 1 + \mathbf A)
    (\mathbf 1 - \mathbf A)
    (\mathbf 1 - \mathbf A)^{-1}
    = \mathbf 1
\end{align*}
which means that $\mathbf{B}$ is an orthogonal matrix.
\end{solution}



% \end{multicols*}
\end{document}

