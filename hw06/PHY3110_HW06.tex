%! TeX program    = xelatex
%! TEX-TS program = xelatex
\documentclass[twoside,11pt]{article}
\usepackage[left=1in, right=1in, top=1in, bottom=1in]{geometry}
\usepackage{amsmath}
\usepackage{amssymb}
\usepackage{amsfonts}
\usepackage{mathtools}
\usepackage{amsthm}
\usepackage{fancyhdr}
\usepackage{enumitem}
\usepackage{siunitx}
\usepackage{booktabs}
\usepackage[hidelinks]{hyperref}
\usepackage{sectsty}
\usepackage{mathrsfs} % mathscr
\usepackage{tikz}
\usepackage{pgfplots}
\usepackage{multicol}
\usepackage{listings}
% \usepackage{amsart}
\usepackage{fontspec}
\usepackage{soul}


% allow H option of figure
\usepackage{float}

% math font (libertine)
\usepackage{libertinus-otf}

% braket
\usepackage{braket}

% tikz library
\usetikzlibrary{decorations,calligraphy,calc,external}
\tikzexternalize
\tikzexternaldisable

% physics
% \usepackage{physics}

% define latin modern font environment
\newcommand{\lms}{\fontfamily{lmss}\selectfont} % Latin Modern Roman
% \newcommand{\lmss}{\fontfamily{lmss}\selectfont} % Latin Modern Sans
% \newcommand{\lmss}{\fontfamily{lmtt}\selectfont} % Latin Modern Mono

% % change mathcal shape
% \usepackage[mathcal]{eucal}


% define math operators
\newcommand{\FF}{\mathbb{F}}
\newcommand{\RR}{\mathbb{R}}
\newcommand{\NN}{\mathbb{N}}
\newcommand{\ZZ}{\mathbb{Z}}
\newcommand{\QQ}{\mathbb{Q}}
\newcommand{\XX}{\mathbb{Y}}
\newcommand{\CL}{\mathcal{L}}
% \renewcommand{\d}{\mathrm{d}}
\renewcommand*\d{\mathop{}\!\mathrm{d}}
\DeclareMathOperator*{\argmax}{arg\,max}
\DeclareMathOperator*{\argmin}{arg\,min}
\DeclareMathOperator{\im}{im}
\DeclareMathOperator{\id}{id}
\DeclareMathOperator{\erf}{erf}
\renewcommand{\mod}[1]{\ (\mathrm{mod}\ #1)}

% section font style
\sectionfont{\sffamily\Large}
\subsectionfont{\sffamily\normalsize}
\subsubsectionfont{\bf}

% line spreading and break
\hyphenpenalty=5000
\tolerance=20
\setlength{\parindent}{0em}
\setlength\parskip{0.5em}
\allowdisplaybreaks
\linespread{0.9}

% enumerate settings
% no break before enumerate
\setlist[enumerate]{itemsep=2pt,topsep=2pt}

% theorem
% latex theorem
% definition style
\theoremstyle{definition}
\newtheorem{theorem}{\lms Theorem}[subsection]
\newtheorem{axiom}{\lms Axiom}[section]
\newtheorem{definition}{\lms Definition}[section]
\newtheorem{example}{\lms Example}[section]
\newtheorem{question}{\lms Question}[section]
\newtheorem{exercise}{\lms Exercise}[section]
\newtheorem*{exercise*}{\lms Exercise}
\newtheorem{lemma}{\lms Lemma}[section]
\newtheorem{proposition}{\lms Proposition}[section]
\newtheorem{corollary}{\lms Corollary}[section]
\newtheorem*{theorem*}{\lms Theorem}
\newtheorem{problem}{\lms Problem}
% remark style
\theoremstyle{remark}
\newtheorem*{remark}{Remark}
\newtheorem*{solution}{Solution}
\newtheorem*{claim}{Claim}


% paragraph indent
\setlength{\parindent}{0em}
\setlength\parskip{0.5em}

\newcommand\Code{PHY3110 SP23}
\newcommand\Ass{HW05}
\newcommand\name{Haoran Sun}
\newcommand\mail{haoransun@link.cuhk.edu.cn}

\title{{\lms \Code \ \Ass}}
\author{\lms \name \ (\href{mailto:\mail}{\mail})}
\date{\sffamily \today}

\makeatletter
% \let\Title\@title
\let\theauthor\@author
\let\thedate\@date

\fancypagestyle{plain}{%
    \fancyhf{}
    \lhead{\lms \Ass}
    \rhead{\lms \name}
    \rfoot{\lms\thepage}

    % # 页脚自定义
    \fancyfoot[L]{
        \begin{minipage}[c]{0.06\textwidth}
            \includegraphics[height=7.5mm]{logo2.png}
        \end{minipage}
    }
}
\fancypagestyle{title}{%
    \fancyhf{}
    \renewcommand{\headrulewidth}{0pt}
    % \lhead{\Title}
    % \rhead{\theauthor}
    \rfoot{\lms\thepage}

    % # 页脚自定义
    \fancyfoot[L]{
        \begin{minipage}[c]{0.06\textwidth}
            \includegraphics[height=7.5mm]{logo2.png}
        \end{minipage}
    }
}
\fancyfootoffset[L]{0.3cm}

% re-define title format
\makeatletter
\renewcommand{\maketitle}{\bgroup\setlength{\parindent}{0pt}
\begin{flushleft}
  \textbf{\Large\@title}

  \@author
\end{flushleft}\egroup
}
\makeatother

\pagestyle{plain}

% lstlisting settings
\lstset{
    basicstyle=\linespread{0.7}\footnotesize,
    breaklines=true,
    basewidth=0.5em
}


\begin{document}
\maketitle
\thispagestyle{title}
% \begin{multicols*}{2}

% \begin{remark}
%     $V_\epsilon(x)$ is used to denote a $\epsilon$-neighborhood
%     \begin{align*}
%         V_\epsilon(x) = B_\epsilon(x)\setminus\{x\}
%     \end{align*}
% \end{remark}

\begin{problem}
Obtain the general rotation matrix in terms of the Euler angles by performing
an explicit multiplication of the tree successive rotation matrices.
Verify that the matrix multiplication is associative.
\end{problem}
\begin{solution} 
The three rotation matrices are
\begin{align*}
    D &= 
    \begin{bmatrix}
        \cos\phi & \sin\phi & 0\\
        -\sin\phi & \cos\phi & 0\\
        0 & 0 & 1
    \end{bmatrix},~
    C = 
    \begin{bmatrix}
        1 & 0 & 0\\
        0 & \cos\theta & \sin\theta \\
        0 & -\sin\theta & \cos\theta
    \end{bmatrix},~
    B = 
    \begin{bmatrix}
        \cos\psi & \sin\psi & 0\\
        -\sin\psi & \cos\psi & 0\\
        0 & 0 & 1
    \end{bmatrix}
\end{align*}
Hence we have
\begin{align*}
    A &= BCD = 
    \begin{bmatrix}
        \cos\phi\cos\psi - \cos\theta\sin\phi\sin\psi & 
        \cos\psi\sin\phi + \cos\theta\cos\phi\sin\psi &
        \sin\theta\sin\psi\\
        -\cos\theta\cos\psi\sin\phi - \cos\phi\sin\psi &
        \cos\theta\cos\phi\cos\psi - \sin\phi\sin\psi &
        \cos\psi\sin\theta\\
        \sin\theta\sin\phi &
        -\cos\phi\sin\theta &
        \cos\theta
    \end{bmatrix}
\end{align*}
The matrix multiplication is associative.
\end{solution}


\begin{problem}
Consider the rotation in the following order: first rotate around the $x$ axis
by an angle $\theta$, then around $z'$ axis by an angle $\psi$, and finally
around the old $z$ axis by an angle $\phi$.
Does this lead to the same transformation matrix as that in Problem 1?
Do you have an explanation for this?
\end{problem}
\begin{solution} 
Let $x$ denotes the original coordinate system. 
Let $\xi$, $\eta$, and $\zeta$ denote three coordinate systems after each rotation,
and let $\symbfit{\xi}$, $\symbfit{\eta}$, and $\symbfit{\zeta}$ denote their basis
vectors ($\mathbb{R}^{3\times 3}$), respectively.
Use matrices $B$, $C$, and $D$ from Problem 1.
We can know immediately that the change of basis matrices
\begin{align*}
    \mathcal{C}_{\xi x} &= C,~
    \mathcal{C}_{\eta\xi} = B,~
    \mathcal{C}_{\eta x} = BC
\end{align*}
However, it seems not straightforward to derive change of basis matirx related
to $\zeta$.
To do so, let's first derive $\symbfit{\zeta}$.
Note that
\begin{align*}
    \symbfit{\zeta} &= \mathcal{C}_{x\zeta}
    = \begin{bmatrix}
        \cos\phi  & \sin\phi & 0 \\
        -\sin\phi & \cos\phi & 0 \\
        0 & 0 & 1
    \end{bmatrix}\symbfit{\eta}
    = D^T
    \mathcal{C}_{x\eta}
    = D^T
    \mathcal{C}_{x\xi}
    \mathcal{C}_{\xi\eta}
    = D^T C^T B^T
\end{align*}
Hence the change of basis matrix $\mathcal{C}_{x\zeta}$ is
\begin{align*}
    \mathcal{C}_{\zeta x} &= BCD = A
\end{align*}
which is the same matrix as $A$ in the Problem 1.
\end{solution}


\begin{problem}
A particle is thrown up vertically with initial speed $v_0$, reaches a maximum
and falls back to the ground. Show that the Coriolis deflection when it again 
reaches the ground is opposite in direction, and four times greater in magnitude,
than the Coriolis deflection when it is dropped at rest from the same maximum
height.
\end{problem}
\begin{solution} 
The Coriolis force is in the form 
\begin{align*}
    -2m\symbfit{\omega}\times\mathbf{V}_\text{body}
\end{align*}
Only consider the $z$ direction of $\mathbf{V}_\text{body}$, since the 
acceleration is constant ($g$), we have 
\begin{align*}
    [\mathbf{V}_\text{body}]_z &= -gt + C
\end{align*}
Ignore the centrifugal term, for two cases we have $\mathbf{a}=-\mathbf{k}t+\mathbf{c}$
for some constant vector $\mathbf{k}$, hence for two cases
\begin{enumerate}[label=\roman*.]
\item Overall time $2T$ ($v_0 = gT$)
\begin{align*}
    \mathbf{a} &= \mathbf{k}T - \mathbf{k}t,~
    \mathbf{v} = \mathbf{k}Tt - \frac{1}{2}\mathbf{k}t^2 + \frac{1}{2}\mathbf k T^2,~
    \mathbf{x} = \frac{1}{2}kTt^2 - \frac{1}{6}\mathbf{k}T^3 + \frac{1}{2} \mathbf kT^2 t\\
    \Rightarrow \left. \mathbf{x}\right|_0^{2T} &= 
    \frac{5}{6}\mathbf k T^2
\end{align*}

\item Overall time $T$
\begin{align*}
    \mathbf{a} &= -\mathbf k t,~
    \mathbf{v} = \frac{1}{2}\mathbf{k}t^2,~
    \mathbf{x} = \frac{1}{6}\mathbf{k}t^3\\
    \Rightarrow \left. \mathbf{x}\right|_0^{T} &= 
    -\frac{1}{6}\mathbf k T^2
\end{align*}
Hence the first case has four times greater in magnitude.

\end{enumerate}
\end{solution}


\begin{problem} 
Prove that for a general rigid body motion about a fixed point, the kinetic
energy $T$ satisfies
\begin{align}
    \frac{\d T}{\d t} &= \symbfit{\omega}\cdot \mathbf{N}
\end{align}
\end{problem}
\begin{solution} 
For any point $i$ in the rigid body rotating around one point,
one should have
\begin{align*}
    \d\mathbf r_i &= \symbfit\omega\times\mathbf r_i\d t
\end{align*}
Hence
\begin{align*}
    \frac{\d}{\d t}\sum\frac{1}{2}m_i\mathbf{v}_i^2
    &= \sum m_i\mathbf v_i\cdot\mathbf a_i\\
    &= \sum m_i(\symbf\omega\times\mathbf r_i)\cdot \mathbf a_i\\
    &= \sum\symbfit\omega\cdot(\mathbf r_i\times m_i\mathbf a_i)\\
    &= \symbfit\omega\cdot\sum\mathbf r_i\times m_i\mathbf a_i\\
    &= \symbfit\omega\cdot\mathbf N
\end{align*}
\end{solution}


\begin{problem}
A uniform sphere of mass $M$ and radius $R$ rotates around an axis through its
center of mass.
How it the kinetic energy of the sphere related to the angular velocity?
\end{problem}
\begin{solution}
Let the $z$ axis align onto $\symbfit{\omega}$, then
the kinetic energy is
\begin{align*}
    T &= \iint R^2\sin\theta \d\theta\d\phi\frac{M}{4\pi R^2}
    \frac{1}{2}(\omega R\cos\theta)^2\\
      &= \int_0^{2\pi}\d \phi\int_0^\pi\d\theta
      \frac{MR^2\omega^2}{8\pi}\sin\theta\cos^2\theta\\
      &= \frac{MR^2\omega^2}{6}
\end{align*}
\end{solution}



% \end{multicols*}
\end{document}

