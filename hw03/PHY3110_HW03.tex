\documentclass[twoside,11pt]{article}
\usepackage[left=1in, right=1in, top=1in, bottom=1in]{geometry}
\usepackage{amsmath}
\usepackage{amssymb}
\usepackage{amsfonts}
\usepackage{mathtools}
\usepackage{amsthm}
\usepackage{fancyhdr}
\usepackage{enumitem}
\usepackage{siunitx}
\usepackage{booktabs}
\usepackage[hidelinks]{hyperref}
\usepackage{sectsty}
\usepackage{mathrsfs} % mathscr
\usepackage{tikz}
\usepackage{pgfplots}
\usepackage{multicol}
\usepackage{listings}
% \usepackage{amsart}
\usepackage{fontspec}
\usepackage{soul}


% allow H option of figure
\usepackage{float}

% math font (libertine)
\usepackage{libertinus-otf}

% braket
\usepackage{braket}

% physics
% \usepackage{physics}

% define latin modern font environment
\newcommand{\lms}{\fontfamily{lmss}\selectfont} % Latin Modern Roman
% \newcommand{\lmss}{\fontfamily{lmss}\selectfont} % Latin Modern Sans
% \newcommand{\lmss}{\fontfamily{lmtt}\selectfont} % Latin Modern Mono

% % change mathcal shape
% \usepackage[mathcal]{eucal}


% define math operators
\newcommand{\FF}{\mathbb{F}}
\newcommand{\RR}{\mathbb{R}}
\newcommand{\NN}{\mathbb{N}}
\newcommand{\ZZ}{\mathbb{Z}}
\newcommand{\QQ}{\mathbb{Q}}
\newcommand{\XX}{\mathbb{Y}}
\newcommand{\CL}{\mathcal{L}}
% \renewcommand{\d}{\mathrm{d}}
\renewcommand*\d{\mathop{}\!\mathrm{d}}
\DeclareMathOperator*{\argmax}{arg\,max}
\DeclareMathOperator*{\argmin}{arg\,min}
\DeclareMathOperator{\im}{im}
\DeclareMathOperator{\id}{id}
\DeclareMathOperator{\erf}{erf}
\renewcommand{\mod}[1]{\ (\mathrm{mod}\ #1)}

% section font style
\sectionfont{\sffamily\Large}
\subsectionfont{\sffamily\normalsize}
\subsubsectionfont{\bf}

% line spreading and break
\hyphenpenalty=5000
\tolerance=20
\setlength{\parindent}{0em}
\setlength\parskip{0.5em}
\allowdisplaybreaks
\linespread{0.9}

% enumerate settings
% no break before enumerate
\setlist[enumerate]{itemsep=2pt,topsep=2pt}

% theorem
% latex theorem
% definition style
\theoremstyle{definition}
\newtheorem{theorem}{\lms Theorem}[subsection]
\newtheorem{axiom}{\lms Axiom}[section]
\newtheorem{definition}{\lms Definition}[section]
\newtheorem{example}{\lms Example}[section]
\newtheorem{question}{\lms Question}[section]
\newtheorem{exercise}{\lms Exercise}[section]
\newtheorem*{exercise*}{\lms Exercise}
\newtheorem{lemma}{\lms Lemma}[section]
\newtheorem{proposition}{\lms Proposition}[section]
\newtheorem{corollary}{\lms Corollary}[section]
\newtheorem*{theorem*}{\lms Theorem}
\newtheorem{problem}{\lms Problem}
% remark style
\theoremstyle{remark}
\newtheorem*{remark}{\lms Remark}
\newtheorem*{solution}{\lms Solution}
\newtheorem*{claim}{\lms Claim}


% paragraph indent
\setlength{\parindent}{0em}
\setlength\parskip{0.5em}

\newcommand\Code{PHY3110 FA22}
\newcommand\Ass{HW03}
\newcommand\name{Haoran Sun}
\newcommand\mail{haoransun@link.cuhk.edu.cn}

\title{{\lms \Code \ \Ass}}
\author{\lms \name \ (\href{mailto:\mail}{\mail})}
\date{\sffamily \today}

\makeatletter
% \let\Title\@title
\let\theauthor\@author
\let\thedate\@date

\fancypagestyle{plain}{%
    \fancyhf{}
    \lhead{\lms \Ass}
    \rhead{\lms \name}
    \rfoot{\lms\thepage}

    % # 页脚自定义
    \fancyfoot[L]{
        \begin{minipage}[c]{0.06\textwidth}
            \includegraphics[height=7.5mm]{logo2.png}
        \end{minipage}
    }
}
\fancypagestyle{title}{%
    \fancyhf{}
    \renewcommand{\headrulewidth}{0pt}
    % \lhead{\Title}
    % \rhead{\theauthor}
    \rfoot{\lms\thepage}

    % # 页脚自定义
    \fancyfoot[L]{
        \begin{minipage}[c]{0.06\textwidth}
            \includegraphics[height=7.5mm]{logo2.png}
        \end{minipage}
    }
}
\fancyfootoffset[L]{0.3cm}

% re-define title format
\makeatletter
\renewcommand{\maketitle}{\bgroup\setlength{\parindent}{0pt}
\begin{flushleft}
  \textbf{\Large\@title}

  \@author
\end{flushleft}\egroup
}
\makeatother

\pagestyle{plain}

% lstlisting settings
\lstset{
    basicstyle=\linespread{0.7}\footnotesize,
    breaklines=true,
    basewidth=0.5em
}


\begin{document}
\maketitle
\thispagestyle{title}
% \begin{multicols*}{2}

% \begin{remark}
%     $V_\epsilon(x)$ is used to denote a $\epsilon$-neighborhood
%     \begin{align*}
%         V_\epsilon(x) = B_\epsilon(x)\setminus\{x\}
%     \end{align*}
% \end{remark}

\begin{problem}
It is non-holonomic since we cannot find a function
$f=f(x,y,z)$ and write the constraint into
\begin{align*}
    \d f &= f_x\d x + f_y\d y + f_z\d z = 0
\end{align*}
In this case, we have
\begin{align*}
    (x^2+y^2+z^2+2x)\d x + 
    2y\d y + 
    2z\d z &= 0
\end{align*}
Note that
\begin{align*}
    \frac{\partial}{\partial z}(x^2+y^2+z^2)\neq
    \frac{\partial}{\partial x}(2z)
\end{align*}
which means that we cannot find such $f$.
\end{problem}


\begin{problem}
Minimize the action is equivalent to solve the Lagrange's equation, 
hence
\begin{align*}
    \begin{aligned}
    L(x, \dot{x}, t) &= \frac{1}{2}m\dot{x}^2 + Fx\\
    \frac{\d}{\d t}\frac{\partial L}{\partial \dot{x}} &= 
    m\ddot{x} = 2C\\
    \frac{\partial L}{\partial x} &= F
    \end{aligned}
    \Rightarrow
    \left\{
    \begin{aligned}
        A &= 0\\
        B &= \frac{a}{t_0} - \frac{Ft_0}{2m}\\
        C &= \frac{F}{2m}
    \end{aligned}
    \right.
\end{align*}
\end{problem}


\begin{problem}
Let two generalized coordinate be $x$ and $y$, then the system 
could be written as
\begin{align*}
    L &= \frac{1}{2}m(\dot{x}^2 + \dot{y}^2)
    - mgy,\quad
    y = Ax^2\Rightarrow
    2Ax\d x - \d y = 0
\end{align*}
Hence we can write Lagrange's equation with constraint
\begin{align*}
    \begin{aligned}
        \frac{\d}{\d t}\frac{\partial L}{\partial\dot{x}} - 
        \frac{\partial L}{\partial x} - 2Ax\lambda&= 0\\
        \frac{\d}{\d t}\frac{\partial L}{\partial\dot{y}} - 
        \frac{\partial L}{\partial y} + \lambda &= 0
    \end{aligned}
    \Rightarrow
    \left\{
    \begin{aligned}
        \ddot{x} &= \frac{2Ax\lambda}{m}\\
        \ddot{y} &= -\frac{\lambda}{m} + g
    \end{aligned}
    \right.
\end{align*}
Note that
\begin{align*}
    y &= Ax^2\Rightarrow
    2Ax\dot{x} = \dot{y}\Rightarrow
    \ddot{y} = 2A\dot{x}^2 + 2Ax\ddot{x}
\end{align*}
we can solve $\lambda=\lambda(x,\dot{x},t)$ as
\begin{align*}
    \lambda&= -\frac{2Am\dot{x}^2 + mg}{1 + 4A^2x^2}
\end{align*}
Hence we have constraint force for two coordinates
\begin{align*}
    Q_x &= 2Ax\lambda = -\frac{4A^2mx\dot{x}^2 + Axmg}{1 + 4A^2x^2},~
    Q_y = -\lambda = 
    \frac{2Am\dot{x}^2 + mg}{1 + 4A^2x^2}
\end{align*}
\end{problem}


\begin{problem}
Setup polar coordinates $a$ and $\theta$,
we have constraint $a\geq R+r$.
The Lagrangian could be written as
\begin{align*}
    L &= \frac{1}{2}m(\dot{a}^2 + a^2\dot{\theta}^2) + \frac{1}{2}mr^2(R\dot{\theta}/r)^2 - mga\sin\theta
\end{align*}
Use the constraint $a = r+R\Rightarrow \d a = 0$, we can get Lagrange's equation for the system
\begin{align*}
    m\ddot{a} - ma\dot{\theta}^2 + mg\sin\theta - \lambda &= 0
\end{align*}
Since $\d a = 0\Rightarrow\dot{a} = 0$ and $\ddot{a}=0$, we have
\begin{align*}
    \lambda &= 
    mg\sin\theta - ma\dot{\theta}^2
\end{align*}
Using the conservation law of energy, we have
\begin{align*}
    mg(r+R)(1-\sin\theta) &= 
    \frac{1}{2}m[(r+R)^2 + R^2]\dot{\theta}^2
\end{align*}
Substitute $\dot{\theta}^2$ into the expression of $\sin\theta$, then we have
\begin{align*}
    \lambda &= 
    mg\left[
        \frac{3(r+R)^2+R^2}{(r+R)^2+R^2}\sin\theta - \frac{2(r+R)^2}{(r+R)^2+R^2}
    \right]
\end{align*}
When the hoop falls off the cylinder, we should have $\lambda=0$, hence $\theta$ will be
\begin{align*}
    \theta &= \arcsin\frac{2(r+R)^2}{3(r+R)^2+R^2}
\end{align*}
\end{problem}




% \end{multicols*}
\end{document}

