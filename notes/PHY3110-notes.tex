\documentclass[twoside,11pt]{article}
\usepackage[left=1in, right=1in, top=1in, bottom=1in]{geometry}
\usepackage{amsmath}
\usepackage{amssymb}
\usepackage{amsfonts}
\usepackage{mathtools}
\usepackage{amsthm}
\usepackage{fancyhdr}
\usepackage{enumitem}
\usepackage{siunitx}
\usepackage{booktabs}
\usepackage[hidelinks]{hyperref}
\usepackage{sectsty}
\usepackage{mathrsfs} % mathscr
\usepackage{tikz}
\usepackage{pgfplots}
\usepackage{multicol}
\usepackage{listings}
% \usepackage{amsart}
\usepackage{fontspec}
\usepackage{soul}


% allow H option of figure
\usepackage{float}

% math font (libertine)
\usepackage{libertinus-otf}

% braket
\usepackage{braket}

% Chinese support
\usepackage{xeCJK}

% physics
% \usepackage{physics}

% define latin modern font environment
\newcommand{\lms}{\fontfamily{lmss}\selectfont} % Latin Modern Roman
% \newcommand{\lmss}{\fontfamily{lmss}\selectfont} % Latin Modern Sans
% \newcommand{\lmss}{\fontfamily{lmtt}\selectfont} % Latin Modern Mono

% % change mathcal shape
% \usepackage[mathcal]{eucal}


% define math operators
\newcommand{\FF}{\mathbb{F}}
\newcommand{\RR}{\mathbb{R}}
\newcommand{\NN}{\mathbb{N}}
\newcommand{\ZZ}{\mathbb{Z}}
\newcommand{\QQ}{\mathbb{Q}}
\newcommand{\XX}{\mathbb{Y}}
\newcommand{\CL}{\mathcal{L}}
% \renewcommand{\d}{\mathrm{d}}
\renewcommand*\d{\mathop{}\!\mathrm{d}}
\DeclareMathOperator*{\argmax}{arg\,max}
\DeclareMathOperator*{\argmin}{arg\,min}
\DeclareMathOperator{\im}{im}
\DeclareMathOperator{\id}{id}
\DeclareMathOperator{\erf}{erf}
\renewcommand{\mod}[1]{\ (\mathrm{mod}\ #1)}

% section font style
\sectionfont{\lms\Large}
\subsectionfont{\lms\normalsize}
\subsubsectionfont{\bf}

% line spreading and break
\hyphenpenalty=5000
\tolerance=20
\setlength{\parindent}{0em}
\setlength\parskip{0.5em}
\allowdisplaybreaks
\linespread{0.9}

% enumerate settings
% no break before enumerate
\setlist[enumerate]{itemsep=2pt,topsep=2pt}
\setlist[itemize]{itemsep=2pt,topsep=2pt}

% theorem
% latex theorem
% definition style
\theoremstyle{definition}
\newtheorem{theorem}{Theorem}[subsection]
\newtheorem{axiom}{Axiom}[section]
\newtheorem{definition}{Definition}[section]
\newtheorem{example}{Example}[section]
\newtheorem{question}{Question}[section]
\newtheorem{exercise}{Exercise}[section]
\newtheorem*{exercise*}{Exercise}
\newtheorem{lemma}{Lemma}[section]
\newtheorem{proposition}{Proposition}[section]
\newtheorem{corollary}{Corollary}[section]
\newtheorem*{theorem*}{Theorem}
\newtheorem{problem}{Problem}
% remark style
\theoremstyle{remark}
\newtheorem*{remark}{Remark}
\newtheorem*{solution}{Solution}
\newtheorem*{claim}{Claim}


% paragraph indent
\setlength{\parindent}{0em}
\setlength\parskip{0.5em}

\newcommand\Code{PHY3110 SP23}
\newcommand\Ass{Notes}
\newcommand\name{Haoran Sun}
\newcommand\mail{haoransun@link.cuhk.edu.cn}

\title{{\lms \Code \ \Ass}}
\author{\lms \name \ (\href{mailto:\mail}{\mail})}
\date{\lms \today}

\makeatletter
% \let\Title\@title
\let\theauthor\@author
\let\thedate\@date

\fancypagestyle{plain}{%
    \fancyhf{}
    \lhead{\lms\Ass}
    \rhead{\lms\name}
    \rfoot{\lms\thepage}

    % # 页脚自定义
    \fancyfoot[L]{
        \begin{minipage}[c]{0.06\textwidth}
            \includegraphics[height=7.5mm]{logo2.png}
        \end{minipage}
    }
}
\fancypagestyle{title}{%
    \fancyhf{}
    \renewcommand{\headrulewidth}{0pt}
    % \lhead{\Title}
    % \rhead{\theauthor}
    \rfoot{\lms\thepage}

    % # 页脚自定义
    \fancyfoot[L]{
        \begin{minipage}[c]{0.06\textwidth}
            \includegraphics[height=7.5mm]{logo2.png}
        \end{minipage}
    }
}
\fancyfootoffset[L]{0.3cm}

% re-define title format
\makeatletter
\renewcommand{\maketitle}{\bgroup\setlength{\parindent}{0pt}
\begin{flushleft}
  \textbf{\Large\@title}

  \@author
\end{flushleft}\egroup
}
\makeatother

\pagestyle{plain}

% lstlisting settings
\lstset{
    basicstyle=\linespread{0.7}\footnotesize,
    breaklines=true,
    basewidth=0.5em
}


\begin{document}
\maketitle
\thispagestyle{title}
% \begin{multicols*}{2}

% \begin{remark}
%     $V_\epsilon(x)$ is used to denote a $\epsilon$-neighborhood
%     \begin{align*}
%         V_\epsilon(x) = B_\epsilon(x)\setminus\{x\}
%     \end{align*}
% \end{remark}

% \tableofcontents

\setcounter{section}{-1}
\section{Introduction}
\textbf{Grading:} 30\% homework, 30\% midterm, 40\% final.

\textbf{Textbooks:}
\begin{itemize}
    \item H. Goldstein, C. Poole, J. Safko, Classical Mechanics,
    3rd Edition, Pearson.
    \item J.R. Taylor, Classical Mechanics, University Science Books.
    \item T.W.B. Kibble, F.H. Berkshire, Classical Mechanics, 5th Edition,
    Imperial College Press.
    \item 梁昆淼,力学(下册)理论力学,4th Edition,高等教育出版社.
\end{itemize}

Classical mechanics describe the motion of macroscopic objects, which are
not extremely massive and not extremely fast.

\section{Newtonian Mechanics}
Vectorial quantities of motion: position $\mathbf{r}$, 
velocity $\mathbf{v}$, force $\mathbf{F}$,
momentum $\mathbf{p}=m\mathbf{v}$,
angular momentum $\mathbf{L}=\mathbf{r}\times\mathbf{p}$.
Equations of motion are derived from those vector quantities.

Analytical mechanics uses scalar quantities of motion
\begin{itemize}
    \item Kinetic energy $T = \frac{1}{2}m\mathbf{v}^2$
    \item Potential energy $V = V(\mathbf{r})$
\end{itemize}
Equations of motion are derived from those scalar quantities.
\subsection{Newton's Laws}
Newton's 2\textsuperscript{nd} law
\begin{align}
    \mathbf{F} = \frac{\d \mathbf{p}}{\d t} = m\mathbf{a}
\end{align}
valid in an inertial frame.
Angular momentum $\mathbf{L}$ and torque $\mathbf{N}$ are also related
\begin{align}
    \frac{\d\mathbf{L}}{\d t} &= 
    \frac{\d}{\d t}(\mathbf{r}\times\mathbf{p})
    = \mathbf{r}\times\mathbf{F}
    = \mathbf{N}
\end{align}

Work done by external forces
\begin{align}
    W_{12} &= \int_1^2\mathbf{F}\d\mathbf{s}
    = \int_1^2 m\frac{\d\mathbf{v}}{\d t}\d\mathbf{s}
    = \int_1^2 m\mathbf{v}\d\mathbf{v}
    = \left.\frac{1}{2}m\mathbf{v}^2\right|_1^2
\end{align}

Define a scalar function $V(\mathbf{r})$,
then $\mathbf{F}=-\nabla V(\mathbf{r})$ is a conservative force.
\begin{align}
    \oint \mathbf{F}\d\mathbf{s} = 0
\end{align}

Center of mass of the system
\begin{align}
    \mathbf{R} &= 
    \frac{\sum_i m_i\mathbf{r}_i}{\sum_i m_i}
    =
    \frac{\sum_i m_i\mathbf{r}_i}{M}
\end{align}
Total momentum
\begin{align}
    \mathbf{P} 
    &= 
    \sum_i m_i\mathbf{p}_i
    = M\dot{\mathbf{R}}
\end{align}
Hence $\mathbf{P}$ is conserved if external force $\mathbf{F}^{(e)}$ is
zero.

Total angular momentum
\begin{align*}
    \frac{\d \mathbf{L}}{\d t} &= 
    \frac{\d }{\d t} \sum_i\mathbf{r}_i\times\mathbf{p}_i
    = \sum_i \mathbf{r}_i\times \left(
        \mathbf{F}_i^{(e)} + \sum_j \mathbf{F}_{ij}
    \right)
    = \sum_i\mathbf{r}_i\times\mathbf{F}_i^{(e)}
    + \sum_{ij}\mathbf{r}_i\times\mathbf{F}_{ij}
\end{align*}
Since $\mathbf{r}_{ij}$ parallel to $\mathbf{F}_ij$, then
\begin{align}
    \sum_{ij}\mathbf{r}_i\mathbf{F}_{ij} = 
    \frac{1}{2}\sum_{ij}\mathbf{r}_{ij}\times\mathbf{F}_{ji} = 0
\end{align}
Therefore
\begin{align}
    \frac{\d\mathbf{L}}{\d t} &= \mathbf{N}^{(e)}
\end{align}

Decomposition of the angular momentum
\begin{align}
    \mathbf{L} 
    &= \sum_i\mathbf{r}_i\times\mathbf{p}_i
    = \sum_i(\mathbf{R} + \mathbf{r}_i)\times m_i(\mathbf{V} + \mathbf{v}_i')
    = \sum_i \mathbf{R}\times m_i\mathbf{V}
    + \sum_i\mathbf{r}_i'\times m_i\mathbf{v}_i'
\end{align}

\subsection{Constraints}
Holonomic constraint
\begin{align}
    f(\mathbf{r}_1, \mathbf{r}_2, \dots, \mathbf{r}_N, t) &= 0
\end{align}
Example: rigid body
\begin{align}
    (\mathbf{r}_i-\mathbf{r}_j)^2 - c_{ij}^2 &= 0
\end{align}
Example: non-sliding cylinder
\begin{align*}
    \dot{x} - R\dot{\theta} = 0
    \Rightarrow x - R\theta = \text{const}
\end{align*}
A constraint of the form
\begin{align}
    \sum_i g_i(\mathbf{x}_1,\mathbf{x}_2,\dots,\mathbf{x}_n)\d\mathbf{x}_i = 0
    \Rightarrow
    \d G(\mathbf{x}_1,\dots) = 0
    \Rightarrow
    G(\mathbf{x}_1,\dots) = \text{const}
\end{align}

Non-holonomic constraint: cannot be written in the form of holonomic
constraint.

\subsection{Generalized coordinates}
Suppose we have a $N$-particle system, 
we will have $3N$ DOFs.
With $k$ constraints, we will have $3N-k$ DOFs.
Define $q_1,\dots,q_{3N-k}$ generalized coordinates,
we have
\begin{align}
    \mathbf{r}_i &= \mathbf{r}_i(q_1,\dots, q_{3N-1}, t)
\end{align}

\section{Lagrange Formalism}
\subsection{D'Alembert's Principle}
Hint from the rigid body: internal forces of constraints do not work.

Virtual displacement: $\delta\mathbf{r}_i$
is consistent with the constraints imposed on the system at
a given time
\begin{align}
    \mathbf{r}_i \rightarrow 
    \mathbf{r}_i + \delta\mathbf{r}_i
\end{align}
Consider a system in equilibrium
\begin{align}
    \mathbf{F}_i = 0
    \Rightarrow
    \sum_i\mathbf{F}_i\cdot\delta\mathbf{r}_i = 0
\end{align}
Separate $\mathbf{F}_i = \mathbf{F}_i^(a) + \mathbf{f}_i$
where $\mathbf{f}_i$ is the constraint force.
Hence
\begin{align}
    \sum_i(\mathbf{F}_i^{(a)} + \mathbf{f}_i)
    \cdot\delta\mathbf{r}_i = 0
    \Rightarrow
    \sum_i\mathbf{F}_i^{(a)}\cdot\delta\mathbf{r}_i = 0
\end{align}
For a system moving under external forces
\begin{align}
    \mathbf{F}_i - \dot{\mathbf{p}}_i = 0
    \Rightarrow
    \sum_i(\mathbf{F}_i - \dot{\mathbf{p}}_i)\delta\mathbf{r}_i = 0
    \Rightarrow
    \sum_i(\mathbf{F}_i^{(a)} - \dot{\mathbf{p}}_i)\delta\mathbf{r}_i = 0
\end{align}
For holonomic constraints
\begin{align}
    \mathbf{r}_i = \mathbf{r}_i(q_1,\dots,q_n,t),\quad
    \mathbf{v}_i = \frac{\d\mathbf{r}_i}{\d t}
    = \frac{\partial \mathbf{r}_i}{\partial t}
    + \sum_j\frac{\partial\mathbf{r}_i}{\partial q_j}\dot{q}_j,\quad
    \delta\mathbf{r}_i = \sum_j\frac{\partial\mathbf{r}_i}{\partial q_j}
    \delta q_j
\end{align}
Generalized force $Q_j$
\begin{align}
    \sum_i\mathbf{F}_i\delta\mathbf{r}_i
    &= \sum_{ij}\mathbf{F}_i\frac{\partial\mathbf{r}_i}{\partial q_j}\delta q_j
    = \sum_j Q_j\delta q_j
\end{align}
Then
\begin{align}
    \sum_i\dot{\mathbf{p}}_i\cdot\delta\mathbf{r}_i
    &= \sum_{ij}m_i\ddot{\mathbf{r}}_i
    \cdot\frac{\partial \mathbf{r}_i}{\partial q_j}\delta q_j
    = 
    \sum_{ij}\left[
        \frac{\d}{\d t}
        \left(m_i\dot{\mathbf{r}}_i\cdot\frac{\partial\mathbf{r}_i}{\partial q_j}\right)
        - m_i\dot{\mathbf{r}}_i\frac{\d}{\d t}\frac{\partial\mathbf{r}_i}{\partial q_j}
    \right]\delta q_j\\
    &= 
    \sum_i\left[
        \frac{\d }{\d t}\left(\frac{\partial T}{\partial\dot{q}_j}\right)
        - \frac{\partial T}{\partial q_j}
    \right]\delta q_j
    = \sum_j Q_j\delta q_j\\
\end{align}
Hence $\forall j$ we have
\begin{align}
    \frac{\d }{\d t}\left(\frac{\partial T}{\partial\dot{q}_j}\right)
    - \frac{\partial T}{\partial q_j}
    - Q_j = 0
    \label{lag0}
\end{align}
Let the potential energy $V=V(\mathbf{r}_i,\dots)=V(q_j,\dots)$,
then we have
\begin{align}
    Q_j &= \sum_i\mathbf{F}_i\frac{\partial\mathbf{r}_i}{\partial q_j}
    = \sum_i -\nabla_i V\frac{\partial\mathbf{r}_i}{\partial q_j}
    = -\frac{\partial V}{\partial q_j}
\end{align}
Therefore
\begin{align}
    \frac{\d }{\d t}\left(\frac{\partial (T-V)}{\partial\dot{q}_j}\right)
    - \frac{\partial (T-V)}{\partial q_j}
    - Q_j = 0
\end{align}



% \end{multicols*}
\end{document}

