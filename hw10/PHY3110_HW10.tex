%! TeX program    = xelatex
%! TEX-TS program = xelatex
\documentclass[twoside,11pt]{article}
\usepackage[left=1in, right=1in, top=1in, bottom=1in]{geometry}
\usepackage{amsmath}
\usepackage{amssymb}
\usepackage{amsfonts}
\usepackage{mathtools}
\usepackage{amsthm}
\usepackage{fancyhdr}
\usepackage{enumitem}
\usepackage{siunitx}
\usepackage{booktabs}
\usepackage[hidelinks]{hyperref}
\usepackage{sectsty}
\usepackage{mathrsfs} % mathscr
\usepackage{tikz}
\usepackage{tikz-3dplot}
\usepackage{pgfplots}
\usepackage{multicol}
\usepackage{listings}
% \usepackage{amsart}
\usepackage{fontspec}
\usepackage{soul}


% allow H option of figure
\usepackage{float}

% math font (libertine)
\usepackage{libertinus-otf}

% braket
\usepackage{braket}

% tikz library
\usetikzlibrary{decorations,calligraphy,calc,external,patterns}
\tikzexternalize
\tikzexternaldisable

% physics
% \usepackage{physics}

% define latin modern font environment
\newcommand{\lms}{\fontfamily{lmss}\selectfont} % Latin Modern Roman
% \newcommand{\lmss}{\fontfamily{lmss}\selectfont} % Latin Modern Sans
% \newcommand{\lmss}{\fontfamily{lmtt}\selectfont} % Latin Modern Mono

% % change mathcal shape
% \usepackage[mathcal]{eucal}


% define math operators
\newcommand{\FF}{\mathbb{F}}
\newcommand{\RR}{\mathbb{R}}
\newcommand{\NN}{\mathbb{N}}
\newcommand{\ZZ}{\mathbb{Z}}
\newcommand{\QQ}{\mathbb{Q}}
\newcommand{\XX}{\mathbb{Y}}
\newcommand{\CL}{\mathcal{L}}
% \renewcommand{\d}{\mathrm{d}}
\renewcommand*\d{\mathop{}\!\mathrm{d}}
\DeclareMathOperator*{\argmax}{arg\,max}
\DeclareMathOperator*{\argmin}{arg\,min}
\DeclareMathOperator{\im}{im}
\DeclareMathOperator{\id}{id}
\DeclareMathOperator{\erf}{erf}
\renewcommand{\mod}[1]{\ (\mathrm{mod}\ #1)}

% section font style
\sectionfont{\sffamily\Large}
\subsectionfont{\sffamily\normalsize}
\subsubsectionfont{\bf}

% line spreading and break
\hyphenpenalty=5000
\tolerance=20
\setlength{\parindent}{0em}
\setlength\parskip{0.5em}
\allowdisplaybreaks
\linespread{0.9}

% enumerate settings
% no break before enumerate
\setlist[enumerate]{itemsep=2pt,topsep=2pt}

% theorem
% latex theorem
% definition style
\theoremstyle{definition}
\newtheorem{theorem}{\lms Theorem}[subsection]
\newtheorem{axiom}{\lms Axiom}[section]
\newtheorem{definition}{\lms Definition}[section]
\newtheorem{example}{\lms Example}[section]
\newtheorem{question}{\lms Question}[section]
\newtheorem{exercise}{\lms Exercise}[section]
\newtheorem*{exercise*}{\lms Exercise}
\newtheorem{lemma}{\lms Lemma}[section]
\newtheorem{proposition}{\lms Proposition}[section]
\newtheorem{corollary}{\lms Corollary}[section]
\newtheorem*{theorem*}{\lms Theorem}
\newtheorem{problem}{\lms Problem}
% remark style
\theoremstyle{remark}
\newtheorem*{remark}{Remark}
\newtheorem*{solution}{Solution}
\newtheorem*{claim}{Claim}


% paragraph indent
\setlength{\parindent}{0em}
\setlength\parskip{0.5em}

\newcommand\Code{PHY3110 SP23}
\newcommand\Ass{HW10}
\newcommand\name{Haoran Sun}
\newcommand\mail{haoransun@link.cuhk.edu.cn}

\title{{\lms \Code \ \Ass}}
\author{\lms \name \ (\href{mailto:\mail}{\mail})}
\date{\sffamily \today}

\makeatletter
% \let\Title\@title
\let\theauthor\@author
\let\thedate\@date

\fancypagestyle{plain}{%
    \fancyhf{}
    \lhead{\lms \Ass}
    \rhead{\lms \name}
    \rfoot{\lms\thepage}

    % # 页脚自定义
    \fancyfoot[L]{
        \begin{minipage}[c]{0.06\textwidth}
            \includegraphics[height=7.5mm]{logo2.png}
        \end{minipage}
    }
}
\fancypagestyle{title}{%
    \fancyhf{}
    \renewcommand{\headrulewidth}{0pt}
    % \lhead{\Title}
    % \rhead{\theauthor}
    \rfoot{\lms\thepage}

    % # 页脚自定义
    \fancyfoot[L]{
        \begin{minipage}[c]{0.06\textwidth}
            \includegraphics[height=7.5mm]{logo2.png}
        \end{minipage}
    }
}
\fancyfootoffset[L]{0.3cm}

% re-define title format
\makeatletter
\renewcommand{\maketitle}{\bgroup\setlength{\parindent}{0pt}
\begin{flushleft}
  \textbf{\Large\@title}

  \@author
\end{flushleft}\egroup
}
\makeatother

\pagestyle{plain}

% lstlisting settings
\lstset{
    basicstyle=\linespread{0.7}\footnotesize,
    breaklines=true,
    basewidth=0.5em
}


\begin{document}
\maketitle
\thispagestyle{title}
% \begin{multicols*}{2}

\begin{problem}
The Lagrangian for a system can be written as
\begin{align}
    L &= a\dot x^2 + b \frac{\dot y}{x} + c\dot x\dot y
    + fy^2\dot x\dot z + g\dot y - k\sqrt{x^2 + y^2}
\end{align}
where $a$, $b$, $c$, $f$, $g$, $k$ are constants.
What is the Hamiltonian?
What quantities are conserved?
\end{problem}

\begin{solution}
The canonical momentum reads
\begin{align*}
    p_x &= 2a\dot x + c\dot y + fy^2\dot z\\
    p_y &= \frac{b}{x} + c\dot x + g\\
    p_z &= fy^2 x
\end{align*}
The Hamiltonian is 
\begin{align*}
    H &= \dot xp_x + \dot yp_y + \dot zp_z - L\\
      &= a\dot x^2 + c\dot x\dot y + fy^2\dot x\dot z + k\sqrt{x^2 + y^2}\\
      &= \dot xp_x - a\dot x^2\\
      &= \frac{p_z}{fy^2}\left( 
          p_x - a \frac{p_z}{fy^2}
      \right) + k\sqrt{x^2 + y^2}
\end{align*}
$H$ is conserved since $\partial L/\partial t = 0$.
Also, $p_z$ is conserved since $z$ is cyclic.
Moreover, $y$ is conserved since $H$ not depend on $p_y$.
\end{solution}



\begin{problem}
For a given Lagrangian
\begin{align}
    L &= \dot q_1^2 + \frac{\dot q_2^2}{a + bq_1^2}
    + k_1q_1^2 + k_2\dot q_1\dot q_2
\end{align}
with $a$, $b$, $k_1$, $k_2$ being constants,
find the equations of motion in the Hamiltonian formulation.
\end{problem}

\begin{solution}
From the Lagrangian we can obtain that
\begin{align*}
    p_1 &= 2\dot q_1 + k_2\dot q_2\\
    p_2 &= k_2\dot q_1 + \frac{2}{a + bq_1^2}\dot q_2
\end{align*}
Hence
\begin{align*}
    \dot q_1 &= \frac{2p_1 - (a+bq_1^2)k_2p_2}{4 - k_2^2(a+bq_1^2)}\\
    \dot q_2 &= \frac{(2p_2 - k_2p_1)(a + bq_1^2)}{4 - k_2(a + bq_1^2)}
\end{align*}
And 
\begin{align*}
    \dot p_1 &= 2k_1q_1 - \frac{2bq_1}{a + bq_1^2}\dot q_2^2
    = 2k_1q_1 - \frac{2bq_1}{a + bq_q^2}\left[ 
        \frac{(2p_2 - k_2p_1)(a + bq_1^2)}{4 - k_2(a + bq_1^2)}
    \right]^2
    \\
    \dot p_2 &= 0
\end{align*}
\end{solution}



\begin{problem}
A Hamiltonian of one degree of freedom has the form
\begin{align}
    H &= \frac{p^2}{2a} - bqpe^{-\alpha t} + 
    \frac{ba}{2}q^2e^{-\alpha t}(\alpha + be^{-\alpha t})
    + \frac{kq^2}{2}
\end{align}
where $a$, $b$, $\alpha$, $k$ are constants.
\begin{enumerate}[label=\alph*)]
    \item Find the Lagrangian corresponding to this Hamiltonian.
    \item Find an equivalent Lagrangian that is not explicitly depend
        on time.
    \item What is the Hamiltonian corresponding to the second Lagrangian,
        and what is the relationship between the two Hamiltonians?
\end{enumerate}
\end{problem}

\begin{solution}~
\begin{enumerate}[label=\alph*)]
\item We can find the expression of $\dot q$
\begin{align*}
    \dot q &= \frac{p}{a} - bqe^{-\alpha t},~
    p = a\dot q + abe^{-\alpha t}
\end{align*}
Hence
\begin{align*}
    L &= p\dot q - H\\
      &= \frac{p^2}{2a} - \frac{ab}{2}q^2e^{-\alpha t}(\alpha + be^{-\alpha t})
      - \frac{kq^2}{2}\\
      &= \frac{1}{2}a\dot q^2 - \frac{1}{2}kq^2
      - \frac{\d }{\d t} abq^2e^{-\alpha t}
\end{align*}

\item Since $L' = L + \d F/\d t$ is equivalent to $L$, we have a $L'$ 
not explicitly depend on time
\begin{align*}
    L' &= \frac{1}{2}a\dot q^2 - \frac{1}{2}kq^2
\end{align*}

\item The Hamiltonian corresponds to $L'$ is
\begin{align*}
    H' &= \frac{1}{2}q\dot q^2 - \frac{1}{2}kq^2
    = \frac{p^2}{2a} + \frac{1}{2}kq^2
\end{align*}
The relationship between two momentums are
\begin{align*}
    p &= p' - \frac{\partial}{\partial\dot q}\frac{\d F}{\d t}
\end{align*}
Since $H=p\dot q - L$, we have
\begin{align*}
    H &= pq - L
    = \left( 
        p' - \frac{\partial}{\partial\dot q}
        \frac{\d F}{\d t}
    \right)\dot q - L' + \frac{\d F}{\d t}
    = H' + \frac{\d F}{\d t} - \dot q \frac{\partial}{\partial\dot q}
    \frac{\d F}{\d t}
\end{align*}
where $F = abq^2e^{-\alpha t}$ in this case.

\end{enumerate}
\end{solution}



\begin{problem}~
\begin{enumerate}[label=\alph*)]
\item The Lagrangian for a system with one degree of freedom reads
\begin{align}
    L &= \frac{m}{2}(\dot q^2\sin^2\omega t + \dot qq\omega\sin2\omega t
    + q^2\omega^2)
\end{align}
What is the corresponding Hamilton? Is it conserved?

\item Introduce a new coordinate defined by $Q=q\sin\omega t$.
Find the Lagrangian in terms of the new coordinate and the corresponding
Hamiltonian. Is $H$ conserved?
\end{enumerate}
\end{problem}

\begin{solution}~
\begin{enumerate}[label=\alph*)]
\item The momentum and the velocity in terms of coordinate and momentum are
\begin{align*}
    p &= m\sin^2\omega t\dot q + \frac{m}{2}q\omega\sin\omega t,~
    \dot q = \frac{1}{m\sin^2\omega t}\left(p - \frac{m}{2}q\omega\sin 2\omega t\right)
\end{align*}
Hence we have
\begin{align*}
    H &= p\dot q - L
    = \frac{m}{2}\sin^2\omega t\dot q^2 - \frac{m}{2}q^2\omega^2
    = \frac{1}{2m\sin^2\omega t}\left( 
        p - \frac{m}{2}q\omega\sin 2\omega t
    \right)^2 - \frac{m}{2}q^2\omega^2
\end{align*}
It is not conserved since $\partial L/\partial t\neq 0$.


\item We can derive
\begin{align*}
    q^2\sin^2\omega t &= \dot Q^2 + \omega^2Q^2\cot^2\omega t - 2Q\dot Q\omega
    \cot\omega t\\
    q\dot q\omega\sin2\omega t &= 
    2\dot QQ\omega\cot\omega t - 2Q^2\omega^2\cot^2\omega t\\
    q^2\omega^2 &= \frac{Q^2\omega^2}{\sin\omega t}
\end{align*}
Hence the new Lagrangian and Hamiltonian are
\begin{align*}
    L &= \frac{m}{2}(\dot Q^2 + Q^2\omega^2),~
    H  = \frac{m}{2}(P^2 - Q^2\omega^2)
\end{align*}
The new Hamiltonian is conserved since the new Lagrangian does not depend on time specifically.

\end{enumerate}
\end{solution}

% \end{multicols*}
\end{document}

