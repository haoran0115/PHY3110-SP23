%! TeX program    = xelatex
%! TEX-TS program = xelatex
\documentclass[twoside,11pt]{article}
\usepackage[left=1in, right=1in, top=1in, bottom=1in]{geometry}
\usepackage{amsmath}
\usepackage{amssymb}
\usepackage{amsfonts}
\usepackage{mathtools}
\usepackage{amsthm}
\usepackage{fancyhdr}
\usepackage{enumitem}
\usepackage{siunitx}
\usepackage{booktabs}
\usepackage[hidelinks]{hyperref}
\usepackage{sectsty}
\usepackage{mathrsfs} % mathscr
\usepackage{tikz}
\usepackage{tikz-3dplot}
\usepackage{pgfplots}
\usepackage{multicol}
\usepackage{listings}
% \usepackage{amsart}
\usepackage{fontspec}
\usepackage{soul}


% allow H option of figure
\usepackage{float}

% math font (libertine)
\usepackage{libertinus-otf}

% braket
\usepackage{braket}

% tikz library
\usetikzlibrary{decorations,calligraphy,calc,external,patterns}
\tikzexternalize
\tikzexternaldisable

% physics
% \usepackage{physics}

% define latin modern font environment
\newcommand{\lms}{\fontfamily{lmss}\selectfont} % Latin Modern Roman
% \newcommand{\lmss}{\fontfamily{lmss}\selectfont} % Latin Modern Sans
% \newcommand{\lmss}{\fontfamily{lmtt}\selectfont} % Latin Modern Mono

% % change mathcal shape
% \usepackage[mathcal]{eucal}


% define math operators
\newcommand{\FF}{\mathbb{F}}
\newcommand{\RR}{\mathbb{R}}
\newcommand{\NN}{\mathbb{N}}
\newcommand{\ZZ}{\mathbb{Z}}
\newcommand{\QQ}{\mathbb{Q}}
\newcommand{\XX}{\mathbb{Y}}
\newcommand{\CL}{\mathcal{L}}
% \renewcommand{\d}{\mathrm{d}}
\renewcommand*\d{\mathop{}\!\mathrm{d}}
\DeclareMathOperator*{\argmax}{arg\,max}
\DeclareMathOperator*{\argmin}{arg\,min}
\DeclareMathOperator{\im}{im}
\DeclareMathOperator{\id}{id}
\DeclareMathOperator{\erf}{erf}
\renewcommand{\mod}[1]{\ (\mathrm{mod}\ #1)}

% section font style
\sectionfont{\sffamily\Large}
\subsectionfont{\sffamily\normalsize}
\subsubsectionfont{\bf}

% line spreading and break
\hyphenpenalty=5000
\tolerance=20
\setlength{\parindent}{0em}
\setlength\parskip{0.5em}
\allowdisplaybreaks
\linespread{0.9}

% enumerate settings
% no break before enumerate
\setlist[enumerate]{itemsep=2pt,topsep=2pt}

% theorem
% latex theorem
% definition style
\theoremstyle{definition}
\newtheorem{theorem}{\lms Theorem}[subsection]
\newtheorem{axiom}{\lms Axiom}[section]
\newtheorem{definition}{\lms Definition}[section]
\newtheorem{example}{\lms Example}[section]
\newtheorem{question}{\lms Question}[section]
\newtheorem{exercise}{\lms Exercise}[section]
\newtheorem*{exercise*}{\lms Exercise}
\newtheorem{lemma}{\lms Lemma}[section]
\newtheorem{proposition}{\lms Proposition}[section]
\newtheorem{corollary}{\lms Corollary}[section]
\newtheorem*{theorem*}{\lms Theorem}
\newtheorem{problem}{\lms Problem}
% remark style
\theoremstyle{remark}
\newtheorem*{remark}{Remark}
\newtheorem*{solution}{Solution}
\newtheorem*{claim}{Claim}


% paragraph indent
\setlength{\parindent}{0em}
\setlength\parskip{0.5em}

\newcommand\Code{PHY3110 SP23}
\newcommand\Ass{HW09}
\newcommand\name{Haoran Sun}
\newcommand\mail{haoransun@link.cuhk.edu.cn}

\title{{\lms \Code \ \Ass}}
\author{\lms \name \ (\href{mailto:\mail}{\mail})}
\date{\sffamily \today}

\makeatletter
% \let\Title\@title
\let\theauthor\@author
\let\thedate\@date

\fancypagestyle{plain}{%
    \fancyhf{}
    \lhead{\lms \Ass}
    \rhead{\lms \name}
    \rfoot{\lms\thepage}

    % # 页脚自定义
    \fancyfoot[L]{
        \begin{minipage}[c]{0.06\textwidth}
            \includegraphics[height=7.5mm]{logo2.png}
        \end{minipage}
    }
}
\fancypagestyle{title}{%
    \fancyhf{}
    \renewcommand{\headrulewidth}{0pt}
    % \lhead{\Title}
    % \rhead{\theauthor}
    \rfoot{\lms\thepage}

    % # 页脚自定义
    \fancyfoot[L]{
        \begin{minipage}[c]{0.06\textwidth}
            \includegraphics[height=7.5mm]{logo2.png}
        \end{minipage}
    }
}
\fancyfootoffset[L]{0.3cm}

% re-define title format
\makeatletter
\renewcommand{\maketitle}{\bgroup\setlength{\parindent}{0pt}
\begin{flushleft}
  \textbf{\Large\@title}

  \@author
\end{flushleft}\egroup
}
\makeatother

\pagestyle{plain}

% lstlisting settings
\lstset{
    basicstyle=\linespread{0.7}\footnotesize,
    breaklines=true,
    basewidth=0.5em
}


\begin{document}
\maketitle
\thispagestyle{title}
% \begin{multicols*}{2}

\begin{problem}
A mass particle moves in a constant vertical gravitational field along the curve
defined by $y=ax^4$, where $y$ is the vertical direction.
Find the equation of motion for small oscillations around the equilibrium position.
\end{problem}
\begin{solution}
Let $L=L(x,\dot x, t)$.
The kinetic energy $T$ is
\begin{align*}
    T &= \frac{1}{2}(\dot x^2 + \dot y^2) = \frac{1}{2}(\dot x^2 + 16a^2x^6\dot x^2)
    = \frac{1}{2}m(1 + 16a^2x^6) \dot x^2
\end{align*}
And the potential energy is
\begin{align*}
    V &= mgy = mgax^4
\end{align*}
Then $V$ has only one equilibrium point at $x_0=0$.
Let $\eta = x-x_0$, around $0$ we have
\begin{align*}
    T &= \frac{1}{2}m \dot \eta^2, ~V= mga\eta^4\\
    L &= T - V = \frac{1}{2}m\dot\eta^2 - mga\eta^4
\end{align*}
Hence the EOM is 
\begin{align*}
    m\ddot \eta + mga \eta &= 0
\end{align*}
\end{solution}




\begin{problem}
Find the normal modes of vibration for a system described by the following kinetic
and potential energy:
\begin{align*}
    T &= \frac{1}{2}mR^2 (\dot\theta_1^2 + \dot\theta_2^2 + \dot\theta_3^2),~
    V  = \frac{1}{2}kR^2[
    (\theta_1 - \theta_2)^2 + 
    (\theta_2 - \theta_3)^2 +
    (\theta_3 - \theta_1)^2
    ]
\end{align*}
\end{problem}
\begin{solution}
Note that
\begin{align*}
    \mathbb{T} &= mR^2\mathbb{I},~
    \mathbb{V}  = kR^2
    \begin{bmatrix}
        2 & -1 & -1\\
        -1 & 2 & -1\\
        -1 & -1 & 2
    \end{bmatrix}
\end{align*}
Then we have the eigenvalue equation
\begin{align*}
    \omega^2\mathbb{T}\mathbf{C} &=
    \mathbb{V}\mathbf{C}
\end{align*}
Easy to solve that there are three modes
\begin{align*}
    \omega_1 &=0,~\mathbf{C}_1 = \frac{1}{\sqrt{3mR^2}}
    \begin{bmatrix}
        1 \\ 1 \\ 1
    \end{bmatrix}\\
    \omega_2 &=\sqrt{3k/m},~\mathbf{C}_2 = \frac{1}{\sqrt{2mR^2}}
    \begin{bmatrix}
        1 \\ -1 \\ 0
    \end{bmatrix}\\
    \omega_3 &=\sqrt{3k/m},~\mathbf{C}_3 = \frac{1}{\sqrt{2mR^2}}
    \begin{bmatrix}
        1 \\ 0 \\ -1
    \end{bmatrix}
\end{align*}
where $\mathbf{C}_1$ corresponds to the translation,
$\mathbf{C}_2$ and $\mathbf{C}_3$ corresponds to the vibration.
\end{solution}




\begin{problem}
Consider a system under small oscillations.
Express its kinetic and potential energy in terms of normal coordiantes.
Try to show that the time average of the kinetic energy is equal to that of the 
potential energy.
\end{problem}
\begin{solution}
For a small oscillation problem
\begin{align*}
    T &= \frac{1}{2}\dot\xi_\alpha^2,~
    V  = \frac{1}{2}\omega_\alpha^2\xi_\alpha^2\\
    L &= \frac{1}{2}\dot\xi_\alpha^2 - \frac{1}{2}\omega_\alpha^2\xi_\alpha^2
\end{align*}
where sum is taken on $\alpha$
Thus we have EOMs
\begin{align*}
    \ddot\xi_\alpha &= -\omega_\alpha^2\xi,~
    \xi = A_\alpha\cos(\omega_\alpha + \varphi_\alpha)
\end{align*}
Hence
\begin{align*}
    T &= \frac{1}{2}A_\alpha^2\omega_\alpha^2\sin^2(\omega_\alpha^2 t + \varphi_\alpha)\\
    V &= \frac{1}{2}A_\alpha^2\omega_\alpha^2\cos^2(\omega_\alpha^2 t + \varphi_\alpha)
\end{align*}
Easy to prove that
\begin{align*}
    \frac{\omega_\alpha}{2\pi}\int_0^{2\pi/\omega_\alpha}
    T_\alpha\d t &= 
    \frac{\omega_\alpha}{2\pi}\int_0^{2\pi/\omega_\alpha}
    V_\alpha\d t
\end{align*}
Hence for enough long time period $\tau$, we have $\overline{T}=\overline{V}$.
\end{solution}




\begin{problem}
A simple pendulum of length $l$ and mass $m$ is attached to a block of mass $2m$,
which can slide on a frictionless surface.
Assume the motion is in the vertical plane, solve the small oscillation problem 
for this system.
\end{problem}
\begin{solution}
Let $x$ denotes the position of the block and $\theta$ be the angle
form by pendulum and the vertical line (counterclockwise).
Then
\begin{align*}
    T &= m\dot x^2 + \frac{1}{2}m
    (l^2\dot\theta^2 + \dot x^2 - 2l\dot x\dot\theta\cos\theta),~
    V  = -\cos\theta mg
\end{align*}
Around the equilibrium point $\theta=0$, we have
\begin{align*}
    T &= m\dot x^2 + \frac{1}{2}m(l\dot\theta - \dot x)^2,~
    V  = \frac{1}{6}mg\dot\theta^2
\end{align*}
Hence
\begin{align*}
    \mathbb{T} &= \begin{bmatrix}
        3m & -lm \\ -lm & ml^2
    \end{bmatrix},~
    \mathbb{V}  = \begin{bmatrix}
        0 & 0 \\ 0 & mg/3
    \end{bmatrix}
\end{align*}
Then we can write an eigenvalue equation
\begin{align*}
    \omega^2\mathbf{C} &= \mathbb{T}^{-1}\mathbb{V}\mathbf{C}
\end{align*}
The solutions are
\begin{align*}
    \omega_1 &= 0,~ \mathbf{C}_1 = \frac{1}{\sqrt{3m}}\begin{bmatrix}
        1\\ 0
    \end{bmatrix}\\
    \omega_2 &= \sqrt{g/2l},~ \mathbf{C}_2 = \frac{1}{\sqrt{ml^2}}\begin{bmatrix}
        l\\ 3
    \end{bmatrix}
\end{align*}
where the first frequency corresponds to the translational motion,
the second frequency corresponds to the oscillation.

\end{solution}

% \end{multicols*}
\end{document}

