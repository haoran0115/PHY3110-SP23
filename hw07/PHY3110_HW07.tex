%! TeX program    = xelatex
%! TEX-TS program = xelatex
\documentclass[twoside,11pt]{article}
\usepackage[left=1in, right=1in, top=1in, bottom=1in]{geometry}
\usepackage{amsmath}
\usepackage{amssymb}
\usepackage{amsfonts}
\usepackage{mathtools}
\usepackage{amsthm}
\usepackage{fancyhdr}
\usepackage{enumitem}
\usepackage{siunitx}
\usepackage{booktabs}
\usepackage[hidelinks]{hyperref}
\usepackage{sectsty}
\usepackage{mathrsfs} % mathscr
\usepackage{tikz}
\usepackage{tikz-3dplot}
\usepackage{pgfplots}
\usepackage{multicol}
\usepackage{listings}
% \usepackage{amsart}
\usepackage{fontspec}
\usepackage{soul}


% allow H option of figure
\usepackage{float}

% math font (libertine)
\usepackage{libertinus-otf}

% braket
\usepackage{braket}

% tikz library
\usetikzlibrary{decorations,calligraphy,calc,external}
\tikzexternalize
\tikzexternaldisable

% physics
% \usepackage{physics}

% define latin modern font environment
\newcommand{\lms}{\fontfamily{lmss}\selectfont} % Latin Modern Roman
% \newcommand{\lmss}{\fontfamily{lmss}\selectfont} % Latin Modern Sans
% \newcommand{\lmss}{\fontfamily{lmtt}\selectfont} % Latin Modern Mono

% % change mathcal shape
% \usepackage[mathcal]{eucal}


% define math operators
\newcommand{\FF}{\mathbb{F}}
\newcommand{\RR}{\mathbb{R}}
\newcommand{\NN}{\mathbb{N}}
\newcommand{\ZZ}{\mathbb{Z}}
\newcommand{\QQ}{\mathbb{Q}}
\newcommand{\XX}{\mathbb{Y}}
\newcommand{\CL}{\mathcal{L}}
% \renewcommand{\d}{\mathrm{d}}
\renewcommand*\d{\mathop{}\!\mathrm{d}}
\DeclareMathOperator*{\argmax}{arg\,max}
\DeclareMathOperator*{\argmin}{arg\,min}
\DeclareMathOperator{\im}{im}
\DeclareMathOperator{\id}{id}
\DeclareMathOperator{\erf}{erf}
\renewcommand{\mod}[1]{\ (\mathrm{mod}\ #1)}

% section font style
\sectionfont{\sffamily\Large}
\subsectionfont{\sffamily\normalsize}
\subsubsectionfont{\bf}

% line spreading and break
\hyphenpenalty=5000
\tolerance=20
\setlength{\parindent}{0em}
\setlength\parskip{0.5em}
\allowdisplaybreaks
\linespread{0.9}

% enumerate settings
% no break before enumerate
\setlist[enumerate]{itemsep=2pt,topsep=2pt}

% theorem
% latex theorem
% definition style
\theoremstyle{definition}
\newtheorem{theorem}{\lms Theorem}[subsection]
\newtheorem{axiom}{\lms Axiom}[section]
\newtheorem{definition}{\lms Definition}[section]
\newtheorem{example}{\lms Example}[section]
\newtheorem{question}{\lms Question}[section]
\newtheorem{exercise}{\lms Exercise}[section]
\newtheorem*{exercise*}{\lms Exercise}
\newtheorem{lemma}{\lms Lemma}[section]
\newtheorem{proposition}{\lms Proposition}[section]
\newtheorem{corollary}{\lms Corollary}[section]
\newtheorem*{theorem*}{\lms Theorem}
\newtheorem{problem}{\lms Problem}
% remark style
\theoremstyle{remark}
\newtheorem*{remark}{Remark}
\newtheorem*{solution}{Solution}
\newtheorem*{claim}{Claim}


% paragraph indent
\setlength{\parindent}{0em}
\setlength\parskip{0.5em}

\newcommand\Code{PHY3110 SP23}
\newcommand\Ass{HW07}
\newcommand\name{Haoran Sun}
\newcommand\mail{haoransun@link.cuhk.edu.cn}

\title{{\lms \Code \ \Ass}}
\author{\lms \name \ (\href{mailto:\mail}{\mail})}
\date{\sffamily \today}

\makeatletter
% \let\Title\@title
\let\theauthor\@author
\let\thedate\@date

\fancypagestyle{plain}{%
    \fancyhf{}
    \lhead{\lms \Ass}
    \rhead{\lms \name}
    \rfoot{\lms\thepage}

    % # 页脚自定义
    \fancyfoot[L]{
        \begin{minipage}[c]{0.06\textwidth}
            \includegraphics[height=7.5mm]{logo2.png}
        \end{minipage}
    }
}
\fancypagestyle{title}{%
    \fancyhf{}
    \renewcommand{\headrulewidth}{0pt}
    % \lhead{\Title}
    % \rhead{\theauthor}
    \rfoot{\lms\thepage}

    % # 页脚自定义
    \fancyfoot[L]{
        \begin{minipage}[c]{0.06\textwidth}
            \includegraphics[height=7.5mm]{logo2.png}
        \end{minipage}
    }
}
\fancyfootoffset[L]{0.3cm}

% re-define title format
\makeatletter
\renewcommand{\maketitle}{\bgroup\setlength{\parindent}{0pt}
\begin{flushleft}
  \textbf{\Large\@title}

  \@author
\end{flushleft}\egroup
}
\makeatother

\pagestyle{plain}

% lstlisting settings
\lstset{
    basicstyle=\linespread{0.7}\footnotesize,
    breaklines=true,
    basewidth=0.5em
}


\begin{document}
\maketitle
\thispagestyle{title}
% \begin{multicols*}{2}

\begin{problem}
Derive Euler's equaiton from the Lagrange equations of motion.
\end{problem}
\begin{solution}
Set up the coordinate system on the rigid body which could diagonalize the
inertia tensor.
Then the kinetic energy is
\begin{align*}
    T &= \frac{1}{2}\sum_i I_i\omega_i^2
\end{align*}
where the angular velocity vector $\symbfit{\omega}$ expressed under the rotated 
coordinates is
\begin{align*}
    [\symbfit{\omega}]_{x'y'z'} &= \begin{bmatrix}
        \dot\phi\sin\theta\sin\psi + \dot\theta\cos\psi \\
        \dot\phi\sin\theta\cos\psi - \dot\theta\sin\psi \\
        \dot\psi + \dot\phi\cos\theta
    \end{bmatrix}
\end{align*}

The Lagrange equation tells that
\begin{align*}
    \frac{\d}{\d t}\frac{\partial L}{\partial\dot\psi}
    -\frac{\partial L}{\partial\psi} &= Q_\psi\\
    \frac{\d}{\d t}\sum_i\frac{\partial T}{\partial\omega_i}\frac{\partial\omega_i}{\partial\dot\psi}
    -\sum_i\frac{\partial T}{\partial\omega_i}\frac{\partial\omega_i}{\partial\psi} &= Q_\psi\\
    I_{z'}\dot \omega_{z'} - (I_{x'} - I_{y'})\omega_{x'}\omega_{y'} &= Q_\psi
\end{align*}

\begin{claim}
The generalized force $Q_\psi$ equals to the torque projected on the rotated
coordinate $z'$, namely
\begin{align*}
    Q_\psi &= N_{z'}
\end{align*}
\end{claim}
\begin{proof}
Since
\begin{align*}
    Q_\psi &= -\frac{\partial V}{\partial\psi}
    = -\nabla V\cdot\frac{\partial \mathbf{r}}{\partial\psi}
    = \mathbf F\cdot\frac{\partial \mathbf{r}}{\partial\psi}
\end{align*}
and
\begin{align*}
    \left[\frac{\partial\mathbf{r}}{\partial \psi}\right]_{x'y'z'} 
    &= A\left[\frac{\partial\mathbf{r}}{\partial \psi}\right]_{xyz}
    = A\frac{\partial}{\partial\psi} A^T[\mathbf{r}]_{x'y'z'}
    = BCD\frac{\partial}{\partial\psi} D^TC^TB^T[\mathbf{r}]_{x'y'z'}\\
    &= B\frac{\partial}{\partial\psi}B^T[r]_{x'y'z'}
    = \begin{bmatrix}
        \cos\psi & \sin\psi & \\
        -\sin\psi & \cos\psi & \\
                  &  & 1
    \end{bmatrix}
    \begin{bmatrix}
        -\sin\psi & -\cos\psi & \\
        \cos\psi & -\sin\psi & \\
                 & & 0
    \end{bmatrix}
    \begin{bmatrix}
        r_{x'} \\ r_{y'} \\ r_{z'}
    \end{bmatrix}
    = \begin{bmatrix}
        -r_{y'}\\ r_{x'}\\ 0
    \end{bmatrix}
\end{align*}
Note that the inner product is invariant under orthonormal transformation
\begin{align*}
    [\mathbf{F}]_{xyz}\cdot\left[\frac{\partial\mathbf{r}}{\partial\psi}\right]_{xyz}
    = 
    [\mathbf{F}]_{x'y'z'}\cdot\left[\frac{\partial\mathbf{r}}{\partial\psi}\right]_{x'y'z'}
\end{align*}
Hence 
\begin{align*}
    Q_\psi &= [\mathbf{F}]_{x'y'z'}\cdot
    \begin{bmatrix}
        -r_{y'}\\ r_{x'}\\ 0
    \end{bmatrix}
    = N_{z'}\qedhere
\end{align*}
\end{proof}

Therefore we get one equation
\begin{align*}
    I_{z'}\dot \omega_{z'} - (I_{x'} - I_{y'})\omega_{x'}\omega_{y'} &= N_{z'}
\end{align*}
Perform an cyclic permutation we can get other two equations
\begin{align*}
    I_{x'}\dot \omega_{x'} - (I_{y'} - I_{z'})\omega_{y'}\omega_{z'} &= N_{x'}\\
    I_{y'}\dot \omega_{y'} - (I_{z'} - I_{x'})\omega_{z'}\omega_{x'} &= N_{y'}
\end{align*}

\end{solution}



\begin{problem}
A uniform rectangular block has mass $M$ and sides $2a$, $2b$ and $2c$.
Find the principal moments of inertia of the block
\begin{enumerate}[label=\roman*)]
\item at its center of mass,
\item at the center of a face of area $4ab$.
\end{enumerate}
Find the moment of inertia of the block
\begin{enumerate}[label=\roman*)]
\item About a space diagonal,
\item about a diagonal of a face of area $4ab$.
\end{enumerate}
\end{problem}
\begin{solution}
For the principal moments of inertia
\begin{enumerate}[label=\roman*)]
\item
Set up the coordinate system similar to the following figure where the origin of the axis
is coincide with the center of the rectangular block, and $x$, $y$, and $z$ axis is parallelly aligned
onto the sides with length $a$, $b$, and $c$, respectively.
\begin{figure}[H]
    \tikzexternalenable
    \centering
    \tdplotsetmaincoords{60}{125}
    \def\a{1.5}
    \def\b{2}
    \def\c{0.8}
    \begin{tikzpicture}[scale=1, tdplot_main_coords]
        \draw (\a, \b, \c) -- (-\a, \b, \c) -- (-\a, -\b, \c) -- (\a, -\b, \c) -- (\a, \b, \c);
        \draw (\a, \b, -\c) -- (-\a, \b, -\c) -- (-\a, -\b, -\c) -- (\a, -\b, -\c) -- (\a, \b, -\c);
        \draw (\a, \b, \c) -- (\a, \b, -\c);
        \draw (-\a, \b, \c) -- (-\a, \b, -\c);
        \draw (\a, -\b, \c) -- (\a, -\b, -\c);
        \draw (-\a, -\b, \c) -- (-\a, -\b, -\c);
        \draw[thick,-latex] (0, 0, 0) -- (3, 0, 0) node [left] {$x$};
        \draw[thick,-latex] (0, 0, 0) -- (0, 3, 0) node [below] {$y$};
        \draw[thick,-latex] (0, 0, 0) -- (0, 0, 3) node [right] {$z$};
    \end{tikzpicture}
    \tikzexternaldisable
\end{figure}
Then $I_{xx}$ equals to
\begin{align*}
    I_{xx} &= \int_{-a}^a\d x \int_{-b}^b\d y \int_{-c}^c\d z~ y^2 + z^2 = \frac{1}{3}M(b^2+c^2)
\end{align*}
and $I_{xy}$ equals to
\begin{align*}
    I_{xy} &= \int_{-a}^a\d x \int_{-b}^b\d y \int_{-c}^c\d z (-xy) = 0
\end{align*}
Hence we can find the inertia tensor $I$ equals to
\begin{align*}
    I &= M\begin{pmatrix}
        (b^2+c^2)/3 & & \\
        & (a^2+c^2)/3 & \\
        & & (a^2+b^2)/3
    \end{pmatrix}
\end{align*}
where the diagonal terms are the principal moments of inertia.

\item Set up a similar coordinate system as in the previous case, then $I_{zz}$
doesn't change since the angular velocity axis doesn't change.
Note that
\begin{align*}
    I_{xx} &= \int_{-a}^a\d x \int_{-b}^b\d y \int_0^{2c}\d z~ y^2 + z^2
    = \frac{1}{3}M(b^2 + 4c^2)\\
    I_{xy} &= \int_{-a}^a\d x \int_{-b}^b\d y \int_0^{2c}\d z~ -xy
    = 0
\end{align*}
Hence the tensor is
\begin{align*}
    I &= M\begin{pmatrix}
        (b^2+4c^2)/3 & & \\
                     & (a^2+4c^2)/3 & \\
                     & & (a^2+b^2)/3
    \end{pmatrix}
\end{align*}
where the diagonal terms are the principal moments of inertia.
    
\end{enumerate}

For the moment inertia
\begin{enumerate}[label=\roman*)]
\item To get the moment inertia along the space diagonal, we can set angular
momentum vector $\symbfit{\omega}$ to
\begin{align*}
    \symbfit{\omega} &= \frac{1}{\sqrt{a^2+b^2+c^2}}\begin{bmatrix}
        a\\ b\\ c
    \end{bmatrix}
\end{align*}
Then the moment of inertia about this axis is
\begin{align*}
    \omega_i I_{ij}\omega_j &=
    \frac{2M}{3}\frac{a^2b^2+b^2c^2+a^2c^2}{a^2+b^2+c^2}
\end{align*}

\item Let the angular momentum vector be
\begin{align*}
    \symbfit\omega &= \frac{1}{\sqrt{a^2+b^2}}\begin{bmatrix}
        a \\ b \\ 0
    \end{bmatrix}
\end{align*}
Then the moment of inertia about this axis is
\begin{align*}
    \omega_i I_{ij}\omega_j &=
    \frac{2M}{3}\frac{a^2b^2 + 2b^2c^2 + 2a^2c^2}{a^2+b^2}
\end{align*}

\end{enumerate}

\end{solution}



\begin{problem}
Consider the torque-free motion of an asymmetric rigid body with one point fixed,
show from Euler equations that $L^2$ and $T$ ($K$ and $T$ are the angular momentum and
kinetic energy) are conserved.
\end{problem}
\begin{solution}
Note that $\mathbf{L}^2$ and its time derivative equals to
\begin{align*}
    \mathbf{L}^2 &= 2L_iL_i\\
    \frac{\d}{\d t}\mathbf{L}^2 &= 2L_i\dot L_i
\end{align*}
The Euler's equation for torque-free motion is
\begin{align*}
    \dot L_i + \epsilon_{ijk}\omega_jL_k &= 0
\end{align*}
Hence
\begin{align*}
    L_i\dot L_i + \epsilon_{ijk}L_i\omega_j L_k &= 0\\
    \Rightarrow
    L_i\dot L_i &= -\epsilon_{ijk}L_iL_k\omega_j
    = \epsilon_{ikj}L_iL_k\omega_j = 0
\end{align*}
which means that $\mathbf{L}^2$ is conserved.

The kinetic energy and its time derivative writes 
\begin{align*}
    T &= \frac{1}{2}\omega_i I_{ij}\omega_j \\
    \frac{\d}{\d t}T &= \frac{1}{2}\dot\omega_i I_{ij}\omega_j
    + \frac{1}{2}\omega_i I_{ij}\dot\omega_j
    = \omega_i I_{ij}\dot\omega_j 
    = \omega_i\dot L_i
\end{align*}
Then from Euler's equation we know that
\begin{align*}
    \omega_i\dot L_i + \epsilon_{ijk}\omega_i\omega_j L_k &= 0\\
    \Rightarrow \omega_i\dot L_i &= -\epsilon_{ijk}\omega_i\omega_j L_k 
    = 0
\end{align*}
which means that $T$ is conserved.
\end{solution}



\begin{problem}
For the axially symmetric rigid body precessing uniformly in the absence of 
torques, find analytical solutions for the Euler angles as a function of time.
\end{problem}
\begin{solution}
Let $I_x=I_y\neq I_z$.
Then the torque-free Euler's equation is
\begin{align*}
    I_x\dot\omega_x - (I_y-I_z)\omega_y\omega_z &= 0\\
    I_y\dot\omega_y - (I_z-I_x)\omega_z\omega_x &= 0\\
    I_z\dot\omega_z &= 0
\end{align*}
Hence we can solve that $\omega_z=\text{const}$.
Define $a=(I_x-I_z)\omega_z/I_x$, then
\begin{align*}
    \begin{aligned}
        \dot\omega_x &= a\omega_y\\
        \dot\omega_y &= -a\omega_x
    \end{aligned}
    \Rightarrow
    \begin{aligned}
        \ddot\omega_x &= -a^2\omega_x\\
        \ddot\omega_y &= -a^2\omega_y
    \end{aligned}
    \Rightarrow
    \begin{aligned}
        \omega_x &= A\sin(at+b)\\
        \omega_y &= A\cos(at+b)
    \end{aligned}
\end{align*}
Hence the angular momentum $\mathbf{L}$ under the body coordinates is
\begin{align*}
    [\mathbf{L}]_b &= I[\symbfit{\omega}]_b
    = \begin{bmatrix}
        I_x A\sin(at+b)\\
        I_x A\cos(at+b)\\
        I_z\omega_z
    \end{bmatrix}
\end{align*}
Since this is a torque-free precession, $\mathbf{L}$ conserved under
the space coordinate.
Let $\mathbf{L}$ align onto the $z$ axis, i.e.
\begin{align*}
    [\mathbf{L}]_s &= 
    \begin{bmatrix}
        0\\ 0\\ L
    \end{bmatrix}
\end{align*}
Hence 
\begin{align*}
    [\mathbf{L}]_b &= A[\mathbf{L}]_s
    = L\begin{bmatrix}
        \sin\theta\sin\psi \\
        \cos\psi\sin\theta\\
        \cos\theta
    \end{bmatrix}
\end{align*}
Therefore we can establish an equality
\begin{align}
    A[\mathbf{L}]_s
    &= L\begin{bmatrix}
        \sin\theta\sin\psi \\
        \cos\psi\sin\theta\\
        \cos\theta
    \end{bmatrix}
    = \begin{bmatrix}
        I_x A\sin(at+b)\\
        I_x A\cos(at+b)\\
        I_z\omega_z
    \end{bmatrix}
    \label{eq:l}
\end{align}
Therefore $\theta=\text{const}$.
Suppose $L>0$, we can also solve get $\psi=at+b$ (else, $\psi=at+b+\pi$).
Recall from the lecture that
\begin{align}
    [\symbfit\omega]_b &= 
    \begin{bmatrix}
        \dot\phi\sin\theta\sin\psi + \dot\theta\cos\psi\\
        \dot\phi\sin\theta\cos\psi - \dot\theta\cos\psi\\
        \dot\psi + \dot\phi\cos\theta
    \end{bmatrix}
    = \begin{bmatrix}
        A\sin(at+b)\\
        A\cos(at+b)\\
        \omega_z
    \end{bmatrix}
    \label{eq:v}
\end{align}
Since $\dot\theta = 0$ and $\psi=ax+b$, we can solve $\dot\phi$
\begin{align*}
    \dot\phi &= \frac{A}{\sin\theta}
\end{align*}
Hence we have the analytical solution for Euler angles
\begin{align*}
    \phi   &= \frac{I_xA}{\sin\theta}t + C \\
    \theta &= \text{const} \\
    \psi   &= at + b
\end{align*}
The solution can be further simplified since we can solve $A$ and $L$ wrt $\theta$
and $\omega_z$, from Equation \ref{eq:l} and \ref{eq:v}
\begin{align*}
    L &= \frac{I_z\omega_z}{\cos\theta},~
    A = \frac{L\sin\theta}{I_x} = \frac{I_z}{I_x}\tan\theta\omega_z
\end{align*}
Thus
\begin{align*}
    \phi   &= \frac{I_z\omega_z}{I_x \cos\theta} \\
    \theta &= \text{const} \\
    \psi   &= \frac{(I_x-I_z)}{I_x}\omega_z t + b
\end{align*}

\end{solution}


% \end{multicols*}
\end{document}

