%! TeX program    = xelatex
%! TEX-TS program = xelatex
\documentclass[twoside,11pt]{article}
\usepackage[left=1in, right=1in, top=1in, bottom=1in]{geometry}
\usepackage{amsmath}
\usepackage{amssymb}
\usepackage{amsfonts}
\usepackage{mathtools}
\usepackage{amsthm}
\usepackage{fancyhdr}
\usepackage{enumitem}
\usepackage{siunitx}
\usepackage{booktabs}
\usepackage[hidelinks]{hyperref}
\usepackage{sectsty}
\usepackage{mathrsfs} % mathscr
\usepackage{tikz}
\usepackage{tikz-3dplot}
\usepackage{pgfplots}
\usepackage{multicol}
\usepackage{listings}
% \usepackage{amsart}
\usepackage{fontspec}
\usepackage{soul}


% allow H option of figure
\usepackage{float}

% math font (libertine)
\usepackage{libertinus-otf}

% braket
\usepackage{braket}

% tikz library
\usetikzlibrary{decorations,calligraphy,calc,external,patterns}
\tikzexternalize
\tikzexternaldisable

% physics
% \usepackage{physics}

% define latin modern font environment
\newcommand{\lms}{\fontfamily{lmss}\selectfont} % Latin Modern Roman
% \newcommand{\lmss}{\fontfamily{lmss}\selectfont} % Latin Modern Sans
% \newcommand{\lmss}{\fontfamily{lmtt}\selectfont} % Latin Modern Mono

% % change mathcal shape
% \usepackage[mathcal]{eucal}


% define math operators
\newcommand{\FF}{\mathbb{F}}
\newcommand{\RR}{\mathbb{R}}
\newcommand{\NN}{\mathbb{N}}
\newcommand{\ZZ}{\mathbb{Z}}
\newcommand{\QQ}{\mathbb{Q}}
\newcommand{\XX}{\mathbb{Y}}
\newcommand{\CL}{\mathcal{L}}
% \renewcommand{\d}{\mathrm{d}}
\renewcommand*\d{\mathop{}\!\mathrm{d}}
\DeclareMathOperator*{\argmax}{arg\,max}
\DeclareMathOperator*{\argmin}{arg\,min}
\DeclareMathOperator{\im}{im}
\DeclareMathOperator{\id}{id}
\DeclareMathOperator{\erf}{erf}
\renewcommand{\mod}[1]{\ (\mathrm{mod}\ #1)}

% section font style
\sectionfont{\sffamily\Large}
\subsectionfont{\sffamily\normalsize}
\subsubsectionfont{\bf}

% line spreading and break
\hyphenpenalty=5000
\tolerance=20
\setlength{\parindent}{0em}
\setlength\parskip{0.5em}
\allowdisplaybreaks
\linespread{0.9}

% enumerate settings
% no break before enumerate
\setlist[enumerate]{itemsep=2pt,topsep=2pt}

% theorem
% latex theorem
% definition style
\theoremstyle{definition}
\newtheorem{theorem}{\lms Theorem}[subsection]
\newtheorem{axiom}{\lms Axiom}[section]
\newtheorem{definition}{\lms Definition}[section]
\newtheorem{example}{\lms Example}[section]
\newtheorem{question}{\lms Question}[section]
\newtheorem{exercise}{\lms Exercise}[section]
\newtheorem*{exercise*}{\lms Exercise}
\newtheorem{lemma}{\lms Lemma}[section]
\newtheorem{proposition}{\lms Proposition}[section]
\newtheorem{corollary}{\lms Corollary}[section]
\newtheorem*{theorem*}{\lms Theorem}
\newtheorem{problem}{\lms Problem}
% remark style
\theoremstyle{remark}
\newtheorem*{remark}{Remark}
\newtheorem*{solution}{Solution}
\newtheorem*{claim}{Claim}


% paragraph indent
\setlength{\parindent}{0em}
\setlength\parskip{0.5em}

\newcommand\Code{PHY3110 SP23}
\newcommand\Ass{HW08}
\newcommand\name{Haoran Sun}
\newcommand\mail{haoransun@link.cuhk.edu.cn}

\title{{\lms \Code \ \Ass}}
\author{\lms \name \ (\href{mailto:\mail}{\mail})}
\date{\sffamily \today}

\makeatletter
% \let\Title\@title
\let\theauthor\@author
\let\thedate\@date

\fancypagestyle{plain}{%
    \fancyhf{}
    \lhead{\lms \Ass}
    \rhead{\lms \name}
    \rfoot{\lms\thepage}

    % # 页脚自定义
    \fancyfoot[L]{
        \begin{minipage}[c]{0.06\textwidth}
            \includegraphics[height=7.5mm]{logo2.png}
        \end{minipage}
    }
}
\fancypagestyle{title}{%
    \fancyhf{}
    \renewcommand{\headrulewidth}{0pt}
    % \lhead{\Title}
    % \rhead{\theauthor}
    \rfoot{\lms\thepage}

    % # 页脚自定义
    \fancyfoot[L]{
        \begin{minipage}[c]{0.06\textwidth}
            \includegraphics[height=7.5mm]{logo2.png}
        \end{minipage}
    }
}
\fancyfootoffset[L]{0.3cm}

% re-define title format
\makeatletter
\renewcommand{\maketitle}{\bgroup\setlength{\parindent}{0pt}
\begin{flushleft}
  \textbf{\Large\@title}

  \@author
\end{flushleft}\egroup
}
\makeatother

\pagestyle{plain}

% lstlisting settings
\lstset{
    basicstyle=\linespread{0.7}\footnotesize,
    breaklines=true,
    basewidth=0.5em
}


\begin{document}
\maketitle
\thispagestyle{title}
% \begin{multicols*}{2}

\begin{problem}
Find the principal moments of inertia about the center of mass of a flat rigid
body in the shape of an isosceles triangle with a uniform mass density.
What are the principal axes?
\end{problem}
\begin{solution}
Easy to prove that the EOM of a uniform isosceles triangle is located at the
$2/3$ location of its height,
hence we can set up coordinate system as the following figure
\begin{figure}[H]
    \tikzexternalenable
    \centering
    \tdplotsetmaincoords{60}{125}
    \def\a{1.2}
    \def\b{1}
    \def\c{0.8}
    \begin{tikzpicture}[scale=1.6, tdplot_main_coords]
        \draw[thick,-latex] (-2, 0, 0) -- (3, 0, 0) node [left] {$x$};
        \draw[thick,-latex] (0, -2, 0) -- (0, 3, 0) node [below] {$y$};
        \draw[<->] (2*\a, 0, 0) -- node[midway,fill=white] {$n$} (2*\a, \b, 0);
        \draw[<->] (2*\a, \b, 0) -- node[midway,fill=white] {$2m$} (0, \b, 0);
        \draw[<->] (0, \b, 0) -- node[midway,fill=white] {$m$} (-\a, \b, 0);
        \draw[pattern=north west lines,pattern color=gray] (-\a, -\b, 0) -- (-\a, \b, 0) -- (2*\a, 0, 0) -- (-\a, -\b, 0);
        \draw[thick,-latex] (0, 0,  0) -- (0, 0, 1.5) node [right] {$z$};
        %\path[pattern=crosshatch] (-\a, -\b, 0) -- (-\a, \b, 0) -- (2*\a, 0, 0) -- (-\a, -\b, 0);
    \end{tikzpicture}
    \tikzexternaldisable
\end{figure}
Let then density be $\rho$, then the total mass $M = 3mn\rho$.
Then we can compute the inertia tensor
\begin{align*}
    I_{xx} &= \int_{-m}^{2m}\d x \int_{-n}^n \d y~ y^2\rho
    = \frac{2}{3}Mn^2\\
    I_{yy} &= \int_{-m}^{2m}\d x \int_{-n}^n \d y~ x^2\rho
    = 2Mm^2\\
    I_{xy} &= -\int_{-m}^{2m}\d x \int_{-n}^n \d y~ xy\rho
    = 0\\
    I_{zz} &= \int_{-m}^{2m}\d x \int_{-n}^n \d y~ x^2+y^2\rho
    = \frac{2}{3}M(3m^2 + n^2)\\
    I_{xz} &= I_{yz} = 0
\end{align*}
Hence $x$, $y$ and $z$ axes in the figure are principal axes.
\end{solution}



\begin{problem}
Consider the torque-free motion of an asymmetric rigid body with one point fixed,
show from Euler equations that $L^2$ and $T$ ($K$ and $T$ are the angular momentum and
kinetic energy) are conserved.
\end{problem}
\begin{solution}
Note that $\mathbf{L}^2$ and its time derivative equals to
\begin{align*}
    \mathbf{L}^2 &= 2L_iL_i\\
    \frac{\d}{\d t}\mathbf{L}^2 &= 2L_i\dot L_i
\end{align*}
The Euler's equation for torque-free motion is
\begin{align*}
    \dot L_i + \epsilon_{ijk}\omega_jL_k &= 0
\end{align*}
Hence
\begin{align*}
    L_i\dot L_i + \epsilon_{ijk}L_i\omega_j L_k &= 0\\
    \Rightarrow
    L_i\dot L_i &= -\epsilon_{ijk}L_iL_k\omega_j
    = \epsilon_{ikj}L_iL_k\omega_j = 0
\end{align*}
which means that $\mathbf{L}^2$ is conserved.

The kinetic energy and its time derivative writes 
\begin{align*}
    T &= \frac{1}{2}\omega_i I_{ij}\omega_j \\
    \frac{\d}{\d t}T &= \frac{1}{2}\dot\omega_i I_{ij}\omega_j
    + \frac{1}{2}\omega_i I_{ij}\dot\omega_j
    = \omega_i I_{ij}\dot\omega_j 
    = \omega_i\dot L_i
\end{align*}
Then from Euler's equation we know that
\begin{align*}
    \omega_i\dot L_i + \epsilon_{ijk}\omega_i\omega_j L_k &= 0\\
    \Rightarrow \omega_i\dot L_i &= -\epsilon_{ijk}\omega_i\omega_j L_k 
    = 0
\end{align*}
which means that $T$ is conserved.
\end{solution}



\begin{problem}~
\begin{enumerate}[label=\arabic*)]
\item Express in terms of Euler's angles the constraint equations for a uniform
sphere rolling without slipping on a flat horizontal surface.
Show that they are nonholonomic.

\item Set up the Lagrangian equations for this problem by the method of Lagrange
multipliers.
Show that the translational and rotational parts of the kinetic energy are separately
conserved.
Are there any other constatns of motion?
\end{enumerate}
\end{problem}
\begin{solution}~
\begin{enumerate}[label=\arabic*)]
\item The constraint should be
\begin{align*}
    \symbfit{\omega}\times\mathbf{R} + \mathbf{V} &= 0
\end{align*}
where $\symbfit{\omega}$ is the angular momentum vector,
$\mathbf R = -R e_z$ is the vector pointing vertically to the plane,
the end point is the contact point of the sphere and the plane,
and $\mathbf{V}=\begin{bmatrix} \dot x & \dot y & 0\end{bmatrix}^T$
is the velocity of the COM of the sphere.

Express all terms wrt. spacial coordinate, then
\begin{gather*}
    [\symbfit{\omega}]_s = \begin{bmatrix}
        \dot\psi\sin\theta\sin\psi + \dot\theta\cos\psi\\
        \dot\phi\sin\theta\cos\psi - \dot\theta\sin\psi\\
        \dot\psi + \dot\phi\cos\theta
    \end{bmatrix}\\
    [\symbfit\omega\times\mathbf R]_s + [\mathbf V]_s =
    \begin{bmatrix}
        R(\dot\psi\sin\theta\cos\psi - \dot\theta\sin\phi) + \dot x\\
        R(\dot\psi\sin\theta\sin\psi + \dot\theta\cos\phi) + \dot y\\
        0 
    \end{bmatrix}
    = 0
\end{gather*}
The constraint is non-holonomic since we can verify it using the universal test.
Consider the first equaiton
\begin{align*}
    R\sin\theta\cos\psi\d\psi - R\sin\phi\d\theta + \d x &= 0
\end{align*}
Let $\psi$, $\phi$, and $x$ be three variables in the test, we have
\begin{align*}
    1(R\cos\theta\cos\psi) + R\sin\theta\cos\psi (0) - R\sin\phi (0) \neq 0
\end{align*}
which means the constraint is non-holonomic.

\item The Lagrangian is
\begin{align*}
    L &= \frac{1}{2}m(\dot x^2 + \dot y^2) + \frac{1}{2} I \symbfit{\omega}^2
    = \frac{1}{2}m(\dot x^2 + \dot y^2) + \frac{1}{2}I
    (\dot\phi^2 + \dot\theta^2 + \dot\psi^2 + 2\dot\psi\dot\phi\cos\theta)
\end{align*}
Consider the Lagrange's equation for $\phi$
\begin{align*}
    \frac{\d}{\d t}\frac{\partial L}{\partial\dot\phi} 
    -\frac{\partial L}{\partial\phi}
    -\mu_1\frac{\partial f_1}{\partial\dot\phi}
    -\mu_2\frac{\partial f_2}{\partial\dot\phi} &= 0\\
    \Rightarrow \frac{\d}{\d t} I(\dot\phi + \dot\psi\cos\theta) &= 
    \dot\omega_z = 0
\end{align*}
which means $\omega_z$ is constant.
Note that $\dot x = \omega_y R$, $\dot y = -\omega_x R$,
then
\begin{align*}
    T &= \frac{1}{2}mR^2 (\omega_x^2 + \omega_y^2) + \frac{1}{2}I(\omega_x^2 + \omega_y^2 + \omega_z^2)
    = \frac{1}{2}(mR^2 + I)(\omega_x^2 + \omega_y^2) + \frac{1}{2}I\omega_z^2
\end{align*}
Since $T$ constant and $\omega_z$ is also constant, we have
$\omega_x^2+\omega_y^2$ a constant.
Which means that the translational kinetic energy $m(\dot x^2 + \dot y^2)/2$
and rotational kinetic energy $I\symbfit{\omega}^2/2$ are constant,
respectively.


\end{enumerate}
\end{solution}



\begin{problem}
A bead of mass $m$ is constrained to move on a hoop of radius $R$.
The hoop rotates with constant angular velocity $\omega$ around a diameter of
the hoop, which is a vertical axis (line along which gravity acts).
\begin{enumerate}[label=\arabic*)]
\item Set up the Lagrangian and obtain the equations of motion of the bead.

\item Find the critical angular velocity $\Omega$ below which the bottom of the
hoop provides a stable equilibrium for the bead.

\item Find the stable equilibrium position for $\omega > \Omega$.
\end{enumerate}
\end{problem}
\begin{solution}~
\begin{enumerate}[label=\arabic*)]
\item Set up the generalized coordinate as the following figure
\begin{figure}[H]
    \centering
    \def\r{1}
    \begin{tikzpicture}[scale=1.2]
        \coordinate (O) at (0, 0);
        \coordinate (B) at (0, -\r);
        \coordinate (M) at ($ (0, 0) + (135:\r) $);
        \draw[thick] (O) circle (\r);
        \draw[fill] (O) circle (1pt);
        \draw[fill] (M) node[above left] {$m$} circle (1pt);
        \draw (M) -- (O);
        \draw[dashed] (0, \r) -- (0, -\r);
        \draw[->] (O) + (-90:0.2) arc (-90:-225:0.2) node[midway,left] {$\theta$};
    \end{tikzpicture} 
\end{figure}
Then the Lagrangian and EOM are
\begin{align*}
    L &= \frac{1}{2} mR^2(\dot\theta^2 + \omega^2\sin^2\theta) + mgR\cos\theta\\
    mR^2\ddot\theta &= mR^2\omega^2 - mgR\sin\theta
\end{align*}

\item The effective potential is
\begin{align*}
    V_\text{eff} &= -\frac{1}{2}mR^2\omega^2\sin^2\theta - mgR\cos\theta
\end{align*}
Its first and second-order derivative (Hessian) is
\begin{align*}
    V'_\text{eff}  &= -mR^2\omega^2\sin\theta\cos\theta + mgR\sin\theta\\
    V''_\text{eff} &= -mR^2\omega^2\cos2\theta + mgR\cos\theta
\end{align*}
Note that $V''_\text{eff}(0) = mgR - mR^2\omega^2$.
If $\theta=0$ is a equilibrium point, the sufficient condition is
$V''_\text{eff}(0) > 0$, so the critical angular velocity $\Omega$
can be solved by $V''\text{eff}(0) = 0$.
Hence
\begin{align*}
    \Omega &= \sqrt{\frac{g}{r}}
\end{align*}

\item Suppose $\omega>\Omega$, and let $\sin\theta\geq 0$, then
we can solve the equilibrium point by solving $V'_\text{eff}=0$
\begin{align*}
    V'_\text{eff} &= 
    -mR^2\omega^2\sin\theta\cos\theta + mgR\sin\theta = 0\\
    \Rightarrow 
    \cos\theta &= \frac{gR}{R^2\omega^2}
\end{align*}
Hence we have to solutions of $\theta$
\begin{align*}
    \theta &= \begin{cases}
        \arccos\frac{gR}{R^2\omega^2} \\
        2\pi - \arccos\frac{gR}{R^2\omega^2}
    \end{cases}
\end{align*}


\end{enumerate}
\end{solution}


% \end{multicols*}
\end{document}

