\documentclass[twoside,11pt]{article}
\usepackage[left=1in, right=1in, top=1in, bottom=1in]{geometry}
\usepackage{amsmath}
\usepackage{amssymb}
\usepackage{amsfonts}
\usepackage{mathtools}
\usepackage{amsthm}
\usepackage{fancyhdr}
\usepackage{enumitem}
\usepackage{siunitx}
\usepackage{booktabs}
\usepackage[hidelinks]{hyperref}
\usepackage{sectsty}
\usepackage{mathrsfs} % mathscr
\usepackage{tikz}
\usepackage{pgfplots}
\usepackage{multicol}
\usepackage{listings}
% \usepackage{amsart}
\usepackage{fontspec}
\usepackage{soul}


% allow H option of figure
\usepackage{float}

% math font (libertine)
\usepackage{libertinus-otf}

% braket
\usepackage{braket}

% physics
% \usepackage{physics}

% define latin modern font environment
\newcommand{\lms}{\fontfamily{lmss}\selectfont} % Latin Modern Roman
% \newcommand{\lmss}{\fontfamily{lmss}\selectfont} % Latin Modern Sans
% \newcommand{\lmss}{\fontfamily{lmtt}\selectfont} % Latin Modern Mono

% % change mathcal shape
% \usepackage[mathcal]{eucal}


% define math operators
\newcommand{\FF}{\mathbb{F}}
\newcommand{\RR}{\mathbb{R}}
\newcommand{\NN}{\mathbb{N}}
\newcommand{\ZZ}{\mathbb{Z}}
\newcommand{\QQ}{\mathbb{Q}}
\newcommand{\XX}{\mathbb{Y}}
\newcommand{\CL}{\mathcal{L}}
% \renewcommand{\d}{\mathrm{d}}
\renewcommand*\d{\mathop{}\!\mathrm{d}}
\DeclareMathOperator*{\argmax}{arg\,max}
\DeclareMathOperator*{\argmin}{arg\,min}
\DeclareMathOperator{\im}{im}
\DeclareMathOperator{\id}{id}
\DeclareMathOperator{\erf}{erf}
\renewcommand{\mod}[1]{\ (\mathrm{mod}\ #1)}

% section font style
\sectionfont{\sffamily\Large}
\subsectionfont{\sffamily\normalsize}
\subsubsectionfont{\bf}

% line spreading and break
\hyphenpenalty=5000
\tolerance=20
\setlength{\parindent}{0em}
\setlength\parskip{0.5em}
\allowdisplaybreaks
\linespread{0.9}

% enumerate settings
% no break before enumerate
\setlist[enumerate]{itemsep=2pt,topsep=2pt}

% theorem
% latex theorem
% definition style
\theoremstyle{definition}
\newtheorem{theorem}{\lms Theorem}[subsection]
\newtheorem{axiom}{\lms Axiom}[section]
\newtheorem{definition}{\lms Definition}[section]
\newtheorem{example}{\lms Example}[section]
\newtheorem{question}{\lms Question}[section]
\newtheorem{exercise}{\lms Exercise}[section]
\newtheorem*{exercise*}{\lms Exercise}
\newtheorem{lemma}{\lms Lemma}[section]
\newtheorem{proposition}{\lms Proposition}[section]
\newtheorem{corollary}{\lms Corollary}[section]
\newtheorem*{theorem*}{\lms Theorem}
\newtheorem{problem}{\lms Problem}
% remark style
\theoremstyle{remark}
\newtheorem*{remark}{\lms Remark}
\newtheorem*{solution}{\lms Solution}
\newtheorem*{claim}{\lms Claim}


% paragraph indent
\setlength{\parindent}{0em}
\setlength\parskip{0.5em}

\newcommand\Code{PHY3110 FA22}
\newcommand\Ass{HW02}
\newcommand\name{Haoran Sun}
\newcommand\mail{haoransun@link.cuhk.edu.cn}

\title{{\lms \Code \ \Ass}}
\author{\lms \name \ (\href{mailto:\mail}{\mail})}
\date{\sffamily \today}

\makeatletter
% \let\Title\@title
\let\theauthor\@author
\let\thedate\@date

\fancypagestyle{plain}{%
    \fancyhf{}
    \lhead{\lms \Ass}
    \rhead{\lms \name}
    \rfoot{\lms\thepage}

    % # 页脚自定义
    \fancyfoot[L]{
        \begin{minipage}[c]{0.06\textwidth}
            \includegraphics[height=7.5mm]{logo2.png}
        \end{minipage}
    }
}
\fancypagestyle{title}{%
    \fancyhf{}
    \renewcommand{\headrulewidth}{0pt}
    % \lhead{\Title}
    % \rhead{\theauthor}
    \rfoot{\lms\thepage}

    % # 页脚自定义
    \fancyfoot[L]{
        \begin{minipage}[c]{0.06\textwidth}
            \includegraphics[height=7.5mm]{logo2.png}
        \end{minipage}
    }
}
\fancyfootoffset[L]{0.3cm}

% re-define title format
\makeatletter
\renewcommand{\maketitle}{\bgroup\setlength{\parindent}{0pt}
\begin{flushleft}
  \textbf{\Large\@title}

  \@author
\end{flushleft}\egroup
}
\makeatother

\pagestyle{plain}

% lstlisting settings
\lstset{
    basicstyle=\linespread{0.7}\footnotesize,
    breaklines=true,
    basewidth=0.5em
}


\begin{document}
\maketitle
\thispagestyle{title}
% \begin{multicols*}{2}

% \begin{remark}
%     $V_\epsilon(x)$ is used to denote a $\epsilon$-neighborhood
%     \begin{align*}
%         V_\epsilon(x) = B_\epsilon(x)\setminus\{x\}
%     \end{align*}
% \end{remark}

\begin{problem}
Let 
\begin{align*}
    \begin{bmatrix}
        x\\y\\z 
    \end{bmatrix}
    &= 
    \begin{bmatrix}
        r \sin\theta\cos\phi \\
        r \sin\theta\sin\phi \\
        r \cos\theta 
    \end{bmatrix}
\end{align*}
and define the following three unit vectors
\begin{align*}
    \hat{r} &= \begin{bmatrix}
        \sin\theta\cos\phi\\
        \sin\theta\sin\phi\\
        \cos\theta
    \end{bmatrix},\quad
    \hat{\phi} = 
    \begin{bmatrix}
        -\sin\phi\\
        \cos\phi\\
        0
    \end{bmatrix},\quad
    \hat{\theta} = 
    \begin{bmatrix}
        \cos\theta\cos\phi\\
        \cos\theta\sin\phi\\
        -\sin\theta
    \end{bmatrix}
\end{align*}
easy to show that $\hat{r}$, $\hat{\phi}$ and $\hat{\theta}$ forms
an orthonormal basis in $\RR^3$.

Note that
\begin{align*}
    \mathbf{F} 
    &= m\ddot{r}
    = m\frac{\d^2}{\d t^2}(r\hat{r})
    = m\frac{\d}{\d t}(\dot{r}\hat{r} + r\dot{\theta}\sin\theta\hat{\phi} + \dot{\theta}r\hat{\theta})
\end{align*}
Note that
\begin{align*}
    \frac{\d}{\d t}\dot{r}\hat{r} 
    &= \ddot{r}\hat{r} + \dot{r}\dot{\phi}\sin\theta\hat{\phi}
    + \dot{r}\dot{\theta}\hat{\theta}\\
    \frac{\d}{\d t}r\dot{\phi}\sin\theta\hat{\phi} &= 
    -r\dot{\phi}^2\sin^2\theta\hat{\theta}
    +(r\ddot{\phi}\sin\theta + \dot{r}\dot{\phi}\sin\theta 
    + r\dot{\phi}\dot{\theta}\cos\theta)\hat{\phi}\\
    - r\dot{\phi}^2\sin\theta\cos\theta\hat{\theta}
    \frac{\d}{\d t}r\dot{\theta}\hat{\theta} &= 
    -r\dot{\theta}^2\hat{r} + r\dot{\phi}\dot{\theta}\cos\theta\hat{\phi}
    + (r\ddot{\theta} + \dot{r}\dot{\theta})\hat{\theta}
\end{align*}
Hence the Newton's 2\textsuperscript{nd} law becomes
\begin{align*}
    \mathbf{F} &= m[
    (\ddot{r} - r\dot{\theta}^2 - r\dot{\phi}^2\sin^2\theta)\hat{r}
    + (r\ddot{\phi} + 2\dot{r}\dot{\phi}\sin\theta + 2r\dot{\theta}\dot{\phi}\cos\theta)\hat{\phi}
    + (r\ddot{\theta} + 2\dot{r}\dot{\theta} - r\dot{\phi}^2\sin\theta\cos\theta)\hat{\theta}]
\end{align*}

\end{problem}


\begin{problem}
Let $L$ be the distance from the hanging point to the center of the ball.
We should have $L>R$.
Define the generalized coordinate $\theta$, we can express the potential as 
\begin{align*}
    V(\theta) &= -\cos\theta\ m_1 gL - 
    [(C-\sqrt{L^2-R^2} - (\theta_0-\theta)R)
    +\cos(\theta_0-\theta)\sqrt{L^2-R^2}
    + R\sin(\theta_0-\theta)
    ]m_2g
\end{align*}
where $\theta_0=\arcsin(R/L)$.
Applying D'Alembert's principle, we would know that
\begin{align*}
    \sum_i Q_i\delta q_i &= 0 \Rightarrow
    -\frac{\partial V}{\partial \theta}\delta\theta = 0
\end{align*}
Note that
\begin{align*}
    \frac{\partial V}{\partial\theta} &= 
    m_1 gL\sin\theta - m_2gR + m_2g\sqrt{L^2-R^2}\sin(\theta-\theta_0) - m_2 gR\cos(\theta-\theta_0)\\
    &= gL[(m_1+m_2)\sin\theta - m_2\sin\theta_0]
\end{align*}
Then 
\begin{align*}
    gL[(m_1+m_2)\sin\theta - m_2\sin\theta_0] &= 0
    \Rightarrow \theta = \arcsin\left(\frac{m_2\sin\theta_0}{m_1+m_2}\right)
\end{align*}
where $\theta\in[0, \theta_0]$.
\end{problem}


\begin{problem}
Define the Lorentz factor $\gamma = 1/\sqrt{1-\dot{x}^2/c^2}$, then
\begin{align*}
    L &= -\frac{m_0c^2}{\gamma} - V
\end{align*}
Since
\begin{align*}
    \frac{\partial L}{\partial x} &= -\frac{\partial V}{\partial x}\\
    \frac{\d}{\d t}\frac{\partial L}{\partial \dot{x}} &= 
    \frac{\d}{\d t}m_0\dot{x}\gamma = m_0\ddot{x}\gamma + m_0\ddot{x}\gamma^3\frac{\dot{x}^2}{c^2}
    = m_0\ddot{x}\gamma^3
\end{align*}
Thus the equation of motion writes
\begin{align*}
    m_0\ddot{x}\gamma^3 + \frac{\partial V}{\partial x} &= 0
\end{align*}

\end{problem}


\begin{problem}
Since for rigid body, the kinetic energy is
\begin{align*}
    K &= \frac{1}{2}mv^2 + \frac{1}{2}I\omega^2
\end{align*}
where $v$ is the velocity of center of mass and $I$ is the rotational inertia at COM.

Define two generalized coordinate: $x$ the horizontal position of center of mass, 
$\theta$ the acute angle formed between the rod and the ground.
For the rod, we have $\mathbf{v}=\dot{x}\hat{x} + l\dot{\theta}\cos\theta~\hat{y}$,
$I = ml^2/3$,
then we can calculate the kinetic and potential energy
\begin{align*}
    K &= \frac{1}{2}m[\dot{x}^2 + (l\dot{\theta}\cos\theta)^2] + \frac{1}{6}ml^2\dot{\theta}^2\\
    V &= \int_0^{2l}\rho\d s = \int_0^{2l\cos\theta}\frac{m}{2l}\frac{1}{\cos\theta} u\d u = mgl\cos\theta
\end{align*}
Hence we can define the Lagrangian and get the equation of motion.
\begin{align*}
    L &= \frac{1}{2}m[\dot{x}^2 + (l\dot{\theta}\cos\theta)^2]
    + \frac{1}{6}ml^2\dot{\theta}^2 - mgl\cos\theta\\
    \frac{\partial L}{\partial x} &= 0\\
    \frac{\d}{\d t}\frac{\partial L}{\partial\dot{x}} &= m\ddot{x}\\
    \frac{\partial L}{\partial \theta} &= 
    -ml^2\dot{\theta}^2\sin\theta\cos\theta
    + mgl\sin\theta\\
    \frac{\d}{\d t}\frac{\partial L}{\partial\dot{\theta}} &= 
    \ddot{\theta}\left(
        ml^2\cos^2\theta + \frac{1}{3}ml^2
    \right)
    - 2ml^2\dot{\theta}^2\sin\theta\cos\theta\\
    \Rightarrow
    &~\ddot{\theta}\left(
        ml^2\cos^2\theta + \frac{1}{3}ml^2
    \right)
    - 2ml^2\dot{\theta}^2\sin\theta\cos\theta
    +ml^2\dot{\theta}^2\sin\theta\cos\theta
    - mgl\sin\theta = 0
\end{align*}
Hence the invariant quantities are $\dot{x}$ and the total energy.

\end{problem}



% \end{multicols*}
\end{document}

