\documentclass[twoside,11pt]{article}
\usepackage[left=1in, right=1in, top=1in, bottom=1in]{geometry}
\usepackage{amsmath}
\usepackage{amssymb}
\usepackage{amsfonts}
\usepackage{mathtools}
\usepackage{amsthm}
\usepackage{fancyhdr}
\usepackage{enumitem}
\usepackage{siunitx}
\usepackage{booktabs}
\usepackage[hidelinks]{hyperref}
\usepackage{sectsty}
\usepackage{mathrsfs} % mathscr
\usepackage{tikz}
\usepackage{pgfplots}
\usepackage{multicol}
\usepackage{listings}
% \usepackage{amsart}
\usepackage{fontspec}
\usepackage{soul}


% allow H option of figure
\usepackage{float}

% math font (libertine)
\usepackage{libertinus-otf}

% braket
\usepackage{braket}

% physics
% \usepackage{physics}

% define latin modern font environment
\newcommand{\lms}{\fontfamily{lmss}\selectfont} % Latin Modern Roman
% \newcommand{\lmss}{\fontfamily{lmss}\selectfont} % Latin Modern Sans
% \newcommand{\lmss}{\fontfamily{lmtt}\selectfont} % Latin Modern Mono

% % change mathcal shape
% \usepackage[mathcal]{eucal}


% define math operators
\newcommand{\FF}{\mathbb{F}}
\newcommand{\RR}{\mathbb{R}}
\newcommand{\NN}{\mathbb{N}}
\newcommand{\ZZ}{\mathbb{Z}}
\newcommand{\QQ}{\mathbb{Q}}
\newcommand{\XX}{\mathbb{Y}}
\newcommand{\CL}{\mathcal{L}}
% \renewcommand{\d}{\mathrm{d}}
\renewcommand*\d{\mathop{}\!\mathrm{d}}
\DeclareMathOperator*{\argmax}{arg\,max}
\DeclareMathOperator*{\argmin}{arg\,min}
\DeclareMathOperator{\im}{im}
\DeclareMathOperator{\id}{id}
\DeclareMathOperator{\erf}{erf}
\renewcommand{\mod}[1]{\ (\mathrm{mod}\ #1)}

% section font style
\sectionfont{\sffamily\Large}
\subsectionfont{\sffamily\normalsize}
\subsubsectionfont{\bf}

% line spreading and break
\hyphenpenalty=5000
\tolerance=20
\setlength{\parindent}{0em}
\setlength\parskip{0.5em}
\allowdisplaybreaks
\linespread{0.9}

% enumerate settings
% no break before enumerate
\setlist[enumerate]{itemsep=2pt,topsep=2pt}

% theorem
% latex theorem
% definition style
\theoremstyle{definition}
\newtheorem{theorem}{Theorem}[subsection]
\newtheorem{axiom}{Axiom}[section]
\newtheorem{definition}{Definition}[section]
\newtheorem{example}{Example}[section]
\newtheorem{question}{Question}[section]
\newtheorem{exercise}{Exercise}[section]
\newtheorem*{exercise*}{Exercise}
\newtheorem{lemma}{Lemma}[section]
\newtheorem{proposition}{Proposition}[section]
\newtheorem{corollary}{Corollary}[section]
\newtheorem*{theorem*}{Theorem}
\newtheorem{problem}{Problem}
% remark style
\theoremstyle{remark}
\newtheorem*{remark}{Remark}
\newtheorem*{solution}{Solution}
\newtheorem*{claim}{Claim}


% paragraph indent
\setlength{\parindent}{0em}
\setlength\parskip{0.5em}

\newcommand\Code{PHY3110 FA22}
\newcommand\Ass{HW01}
\newcommand\name{Haoran Sun}
\newcommand\mail{haoransun@link.cuhk.edu.cn}

\title{{\lms \Code \ \Ass}}
\author{\lms \name \ (\href{mailto:\mail}{\mail})}
\date{\sffamily \today}

\makeatletter
% \let\Title\@title
\let\theauthor\@author
\let\thedate\@date

\fancypagestyle{plain}{%
    \fancyhf{}
    \lhead{\lms \Ass}
    \rhead{\lms \name}
    \rfoot{\lms\thepage}

    % # 页脚自定义
    \fancyfoot[L]{
        \begin{minipage}[c]{0.06\textwidth}
            \includegraphics[height=7.5mm]{logo2.png}
        \end{minipage}
    }
}
\fancypagestyle{title}{%
    \fancyhf{}
    \renewcommand{\headrulewidth}{0pt}
    % \lhead{\Title}
    % \rhead{\theauthor}
    \rfoot{\lms\thepage}

    % # 页脚自定义
    \fancyfoot[L]{
        \begin{minipage}[c]{0.06\textwidth}
            \includegraphics[height=7.5mm]{logo2.png}
        \end{minipage}
    }
}
\fancyfootoffset[L]{0.3cm}

% re-define title format
\makeatletter
\renewcommand{\maketitle}{\bgroup\setlength{\parindent}{0pt}
\begin{flushleft}
  \textbf{\Large\@title}

  \@author
\end{flushleft}\egroup
}
\makeatother

\pagestyle{plain}

% lstlisting settings
\lstset{
    basicstyle=\linespread{0.7}\footnotesize,
    breaklines=true,
    basewidth=0.5em
}


\begin{document}
\maketitle
\thispagestyle{title}
% \begin{multicols*}{2}

% \begin{remark}
%     $V_\epsilon(x)$ is used to denote a $\epsilon$-neighborhood
%     \begin{align*}
%         V_\epsilon(x) = B_\epsilon(x)\setminus\{x\}
%     \end{align*}
% \end{remark}

\begin{problem}
Since 
\begin{align*}
    [a\times(b\times c)]_i &= 
    \epsilon_{ijk}a_j (b\times c)_k
    = \epsilon_{ijk}\epsilon_{klm} a_jb_lc_m
    = (\delta_{il}\delta_{jm} - \delta_{im}\delta_{jl})
    a_jb_lc_m
    = \delta_{il}\delta_{jm}a_jb_lc_m
    - \delta_{im}\delta_{jl}a_jb_lc_m\\
    &= b_i(a\cdot c) - c_i(a\cdot b)
\end{align*}
Hence we have $a\times (b\times c) = (a\cdot c)b - (a\cdot b)c$.

Suppose $b=\nabla$, then we should consider the commutation relationship
of $b$ and $c$, i.e., $[b,c]=bc-cb$ might be nonzero.
At that time, the formula might not hold.

\end{problem}

\begin{problem}
Let 
\begin{align*}
    \begin{bmatrix}
        x\\y\\z 
    \end{bmatrix}
    &= 
    \begin{bmatrix}
        r \sin\theta\cos\phi \\
        r \sin\theta\sin\phi \\
        r \cos\theta 
    \end{bmatrix}
\end{align*}
and define the following three unit vectors
\begin{align*}
    \hat{r} &= \begin{bmatrix}
        \sin\theta\cos\phi\\
        \sin\theta\sin\phi\\
        \cos\theta
    \end{bmatrix},\quad
    \hat{\phi} = 
    \begin{bmatrix}
        \cos\phi\\
        \sin\phi\\
        0
    \end{bmatrix},\quad
    \hat{\theta} = 
    \begin{bmatrix}
        \cos\theta\cos\phi\\
        \cos\theta\sin\phi\\
        -\sin\theta
    \end{bmatrix}
\end{align*}
easy to show that $\hat{r}$, $\hat{\phi}$ and $\hat{\theta}$ forms
an orthonormal basis in $\RR^3$.
Then under spherical coordinates we have
\begin{align*}
    \d\mathbf{r} &= 
    \d r\cdot\hat{r} + r\d\theta\cdot\hat{\theta}
    + r\sin\theta\d\phi\cdot\hat{\phi}
\end{align*}
square the equation, we get
\begin{align*}
    \d\mathbf{r}^2 &= 
    \d r^2 + r^2\d\theta^2 + r^2\sin^2\theta\d\phi^2
\end{align*}

\end{problem}

\begin{problem}
Since
\begin{align*}
    \frac{\d}{\d t}\frac{\partial T}{\partial\dot{q}_i} &= 
    \sum_j
    \frac{\partial^2 T}{\partial\dot{q}_i q_j}\dot{q}_j +
    \frac{\partial^2 T}{\partial\dot{q}_i \dot{q}_j}\ddot{q}_j
    + \frac{\partial^2 T}{\partial\dot{q}_i\partial t}\\
    \dot{T} &= \sum_i\frac{\partial T}{\partial q_i}\dot{q}_i
    + \frac{\partial T}{\partial\dot{q}_i}\ddot{q}_i
    + \frac{\partial T}{\partial t}\\
    \frac{\partial\dot{T}}{\partial\dot{q}_i} &=
    \sum_j\frac{\partial^2 T}{\partial\dot{q}_i\partial\dot{q}_j}\ddot{q}_j
    + \frac{\partial^2 T}{\partial\dot{q}_i q_j}\dot{q}_j
    + \frac{\partial^2 T}{\partial\dot{q}_i\partial t}
\end{align*}
Then
\begin{align*}
    \frac{\d}{\d t}\frac{\partial T}{\partial \dot{q}_i}
    + \frac{\partial T}{\partial q_i}
    - \frac{\partial\dot{T}}{\partial\dot{q}_i} 
    = 0
\end{align*}
Add this equation to the Lagrange's equation, then we have
\begin{align*}
    \frac{\partial\dot{T}}{\partial\dot{q}_i}
    - 2\frac{\partial T}{\partial q_i} &= Q_i
\end{align*}

\end{problem}


\begin{problem}\
\begin{enumerate}[label=(\alph*)]
\item The condition should be
\begin{align*}
    \frac{\partial f}{\partial x_i} &= 
    Mg_i
\end{align*}

\item Define function $f(x, y, z) = x^2+y^2+z^2+xy+yz+xz$, then we have
\begin{align*}
    \d f &= 
    (2x+y+z)\d x + (x+2y+z)\d y + (x+y+2z)\d z
\end{align*}
Hence the following constraint is holonomic.
\begin{align*}
    (2x+y+z)\d x + (x+2y+z)\d y + (x+y+2z)\d z = 0
\end{align*}

The second constraint is not holonomic, since it cannot be written 
into full differential form.

\end{enumerate}
\end{problem}


\begin{problem}
Use the generalized coordinates $\theta$ and $x$ to represent the system,
hence we have
(let $k=m\omega^2$)
\begin{align*}
    T &= \frac{1}{2}m[(l+x)^2\dot{\theta}^2 + \dot{x}^2]\\
    V &= -mg(l+x)\cos\theta + \frac{1}{2}m\omega^2 x\\
    L &= T - V
\end{align*}
Therefore
\begin{align*}
    \frac{\partial L}{\partial x} &=
    m(l+x)\dot{\theta}^2 + mg\cos\theta - m\omega^2x\\
    \frac{\d}{\d t}\frac{\partial L}{\partial \dot{x}} &= 
    m\ddot{x}\\
    \frac{\partial L}{\partial\theta} &=
    -mg(l+x)\sin\theta\\
    \frac{\d}{\d t}\frac{\partial L}{\partial\dot{\theta}} &= 
    m(l+x)^2\ddot{\theta} + 2m(l+x)\dot{\theta}\dot{x}
\end{align*}
Applying the Lagrange's equation we get 
\begin{align*}
    \ddot{x} &=
    (l+x)\dot{\theta}^2 + g\cos\theta - \omega^2 x\\
    \ddot{\theta} &= 
    -\frac{1}{l+x}(g\sin\theta + 2\dot{\theta}\dot{x})
\end{align*}

\end{problem}


% \end{multicols*}
\end{document}

