%! TeX program    = xelatex
%! TEX-TS program = xelatex
\documentclass[twoside,11pt]{article}
\usepackage[left=1in, right=1in, top=1in, bottom=1in]{geometry}
\usepackage{amsmath}
\usepackage{amssymb}
\usepackage{amsfonts}
\usepackage{mathtools}
\usepackage{amsthm}
\usepackage{fancyhdr}
\usepackage{enumitem}
\usepackage{siunitx}
\usepackage{booktabs}
\usepackage[hidelinks]{hyperref}
\usepackage{sectsty}
\usepackage{mathrsfs} % mathscr
\usepackage{tikz}
\usepackage{tikz-3dplot}
\usepackage{pgfplots}
\usepackage{multicol}
\usepackage{listings}
% \usepackage{amsart}
\usepackage{fontspec}
\usepackage{soul}


% allow H option of figure
\usepackage{float}

% math font (libertine)
\usepackage{libertinus-otf}

% braket
\usepackage{braket}

% tikz library
\usetikzlibrary{decorations,calligraphy,calc,external,patterns}
\tikzexternalize
\tikzexternaldisable

% physics
% \usepackage{physics}

% define latin modern font environment
\newcommand{\lms}{\fontfamily{lmss}\selectfont} % Latin Modern Roman
% \newcommand{\lmss}{\fontfamily{lmss}\selectfont} % Latin Modern Sans
% \newcommand{\lmss}{\fontfamily{lmtt}\selectfont} % Latin Modern Mono

% % change mathcal shape
% \usepackage[mathcal]{eucal}


% define math operators
\newcommand{\FF}{\mathbb{F}}
\newcommand{\RR}{\mathbb{R}}
\newcommand{\NN}{\mathbb{N}}
\newcommand{\ZZ}{\mathbb{Z}}
\newcommand{\QQ}{\mathbb{Q}}
\newcommand{\XX}{\mathbb{Y}}
\newcommand{\CL}{\mathcal{L}}
% \renewcommand{\d}{\mathrm{d}}
\renewcommand*\d{\mathop{}\!\mathrm{d}}
\DeclareMathOperator*{\argmax}{arg\,max}
\DeclareMathOperator*{\argmin}{arg\,min}
\DeclareMathOperator{\im}{im}
\DeclareMathOperator{\id}{id}
\DeclareMathOperator{\erf}{erf}
\renewcommand{\mod}[1]{\ (\mathrm{mod}\ #1)}

% section font style
\sectionfont{\sffamily\Large}
\subsectionfont{\sffamily\normalsize}
\subsubsectionfont{\bf}

% line spreading and break
\hyphenpenalty=5000
\tolerance=20
\setlength{\parindent}{0em}
\setlength\parskip{0.5em}
\allowdisplaybreaks
\linespread{0.9}

% enumerate settings
% no break before enumerate
\setlist[enumerate]{itemsep=2pt,topsep=2pt}

% theorem
% latex theorem
% definition style
\theoremstyle{definition}
\newtheorem{theorem}{\lms Theorem}[subsection]
\newtheorem{axiom}{\lms Axiom}[section]
\newtheorem{definition}{\lms Definition}[section]
\newtheorem{example}{\lms Example}[section]
\newtheorem{question}{\lms Question}[section]
\newtheorem{exercise}{\lms Exercise}[section]
\newtheorem*{exercise*}{\lms Exercise}
\newtheorem{lemma}{\lms Lemma}[section]
\newtheorem{proposition}{\lms Proposition}[section]
\newtheorem{corollary}{\lms Corollary}[section]
\newtheorem*{theorem*}{\lms Theorem}
\newtheorem{problem}{\lms Problem}
% remark style
\theoremstyle{remark}
\newtheorem*{remark}{Remark}
\newtheorem*{solution}{Solution}
\newtheorem*{claim}{Claim}


% paragraph indent
\setlength{\parindent}{0em}
\setlength\parskip{0.5em}

\newcommand\Code{PHY3110 SP23}
\newcommand\Ass{HW11}
\newcommand\name{Haoran Sun}
\newcommand\mail{haoransun@link.cuhk.edu.cn}

\title{{\lms \Code \ \Ass}}
\author{\lms \name \ (\href{mailto:\mail}{\mail})}
\date{\sffamily \today}

\makeatletter
% \let\Title\@title
\let\theauthor\@author
\let\thedate\@date

\fancypagestyle{plain}{%
    \fancyhf{}
    \lhead{\lms \Ass}
    \rhead{\lms \name}
    \rfoot{\lms\thepage}

    % # 页脚自定义
    \fancyfoot[L]{
        \begin{minipage}[c]{0.06\textwidth}
            \includegraphics[height=7.5mm]{logo2.png}
        \end{minipage}
    }
}
\fancypagestyle{title}{%
    \fancyhf{}
    \renewcommand{\headrulewidth}{0pt}
    % \lhead{\Title}
    % \rhead{\theauthor}
    \rfoot{\lms\thepage}

    % # 页脚自定义
    \fancyfoot[L]{
        \begin{minipage}[c]{0.06\textwidth}
            \includegraphics[height=7.5mm]{logo2.png}
        \end{minipage}
    }
}
\fancyfootoffset[L]{0.3cm}

% re-define title format
\makeatletter
\renewcommand{\maketitle}{\bgroup\setlength{\parindent}{0pt}
\begin{flushleft}
  \textbf{\Large\@title}

  \@author
\end{flushleft}\egroup
}
\makeatother

\pagestyle{plain}

% lstlisting settings
\lstset{
    basicstyle=\linespread{0.7}\footnotesize,
    breaklines=true,
    basewidth=0.5em
}


\begin{document}
\maketitle
\thispagestyle{title}
% \begin{multicols*}{2}

\begin{problem}
Verify the Jacobi identity for the Poisson brackets.
\end{problem}

\begin{solution} 
Let $A_q = \partial_q A$, then
\begin{align*}
    \{A, \{B, C\}\} 
    &= \{A, B_{q_i}C_{p_i} - B_{p_i}C_{q_i}\}\\
    &= 
    A_{q_j}(B_{q_ip_j}C_{p_i} + B_{q_i}C_{p_ip_j} - B_{p_ip_j}C_{q_i}
    - B_{p_i}C_{q_ip_j})
    -A_{p_j}(B_{q_iq_j}C_{p_i} + B_{q_i}C_{p_iq_j} - B_{p_iq_j}C_{q_i}
    - B_{p_i}C_{q_iq_j})
\end{align*}
Hence $J=\{A, \{B, C\}\} + \{B, \{C, A\}\} + \{C, \{A, B\}\}$ is
\begin{align*}
    J &= 
    A_{q_j}(B_{q_ip_j}C_{p_i} + B_{q_i}C_{p_ip_j} - B_{p_ip_j}C_{q_i}
    - B_{p_i}C_{q_ip_j})
    -A_{p_j}(B_{q_iq_j}C_{p_i} + B_{q_i}C_{p_iq_j} - B_{p_iq_j}C_{q_i}
    - B_{p_i}C_{q_iq_j})\\
      &+
    B_{q_j}(C_{q_ip_j}A_{p_i} + C_{q_i}A_{p_ip_j} - C_{p_ip_j}A_{q_i}
    - C_{p_i}A_{q_ip_j})
    -B_{p_j}(C_{q_iq_j}A_{p_i} + C_{q_i}A_{p_iq_j} - C_{p_iq_j}A_{q_i}
    - C_{p_i}A_{q_iq_j})\\
      &+
    C_{q_j}(A_{q_ip_j}B_{p_i} + A_{q_i}B_{p_ip_j} - A_{p_ip_j}B_{q_i}
    - A_{p_i}B_{q_ip_j})
    -C_{p_j}(A_{q_iq_j}B_{p_i} + A_{q_i}B_{p_iq_j} - A_{p_iq_j}B_{q_i}
    - A_{p_i}B_{q_iq_j})\\
      &= 0
\end{align*}
\end{solution}




\begin{problem}
Show by the use of Poisson brackets that for a one-dimensional harmonic oscillator
there is a constant of motion $u$ defined as
\begin{align}
    u(q,p,t) &= \ln(p + im\omega q) - i\omega t,\quad
    \omega = \sqrt{\frac{k}{m}}
\end{align}
\end{problem}

\begin{solution} 
The Hamiltonian of a one-dimensional harmonic oscillator is 
\begin{align*}
    H(q, p, t) &= \frac{1}{2m}(p^2 + m^2\omega^2 q^2)
\end{align*}
Hence
\begin{align*}
    \frac{\d u}{\d t}
    &= \{u, H\} + \frac{\partial u}{\partial t}\\
    &= \frac{im\omega}{p + im\omega q}\frac{p}{q} - 
    \frac{m\omega^2 q}{p + im\omega q} - i\omega\\
    &= \frac{1}{p + im\omega q}(i\omega p - m\omega^2 q + m\omega^2 q - i\omega p)\\
    &= 0
\end{align*}
\end{solution}




\begin{problem}
Show that the following transformation is canonical ($\alpha$ is a fixed parameter):
\begin{align}
    x &= \frac{1}{\alpha}(\sqrt{2P_1}\sin Q_1 + P_2), \quad 
    p_x = \frac{\alpha}{2}(\sqrt{2P_1}\cos Q_1 - Q_2) \nonumber \\
    y &= \frac{1}{\alpha}(\sqrt{2P_1}\cos Q_1 + Q_2), \quad 
    p_y = -\frac{\alpha}{2}(\sqrt{2P_1}\sin Q_1 - P_2)
\end{align}
Apply this transformation to the problem of a particle of charge $q$ moving
in a plane that is perpendicular to a constant magnetic field $\mathbf{B}$.
Express the Hamiltonian fot this problem in the $(Q_i,P_i)$ coordinates
letting the parameter $\alpha$ take the form
\begin{align}
    \alpha^2 &= \frac{qB}{c}
\end{align}
From this Hamiltonian, obtain the motion of the particle as a function of time.
\end{problem}

\begin{solution} 
To show the transformation is a canonical transformation, consider the Jacobain $\mathbf{J}$
\begin{align*}
    \mathbf{J} &= \frac{\partial [\mathbf{Q} ~~ \mathbf{P}]^T}
    {\partial [\mathbf{q} ~~ \mathbf{p}]^T}\\
               &= \begin{bmatrix}
        (2P_1)^{0.5}\cos Q_1/\alpha & 0 & (2P_1)^{-0.5}\sin Q_1 /\alpha & 1/\alpha\\
        -(2P_1)^{0.5}\sin Q_1/\alpha & 1/\alpha & (2P_1)^{-0.5}\cos Q_1 /\alpha & 0 \\
        -\alpha(2P_1)^{0.5}\sin Q_1 & -\alpha/2 & \alpha (2P_1)^{-0.5}\cos Q_1/2 & 0\\
        -\alpha(2P_1)^{0.5}\cos Q_1 & 0 & \alpha (2P_1)^{-0.5}\cos Q_1/2 & \alpha/2
    \end{bmatrix}
\end{align*}
We can show that
\begin{align*}
    \mathbf{J}
    \begin{bmatrix}
        & -\mathbb{1} \\ \mathbb{1} & 
    \end{bmatrix}
    \mathbf{J}^T &= 
    \begin{bmatrix}
        & -\mathbb{1} \\ \mathbb{1} & 
    \end{bmatrix}
\end{align*}
Hence, it is a cannonical transformation.

To simplify calculation, let $c=1$, then $\alpha^2 = qB$.
Hence
\begin{align*}
    L &= \frac{1}{2}m\mathbf{v}^2 - q\phi + q\mathbf{A}\cdot\mathbf{v},\quad
    H  = \mathbf{v}\cdot \mathbf{p} - L = \frac{1}{2}m\mathbf{v}^2 = \frac{1}{2m}(\mathbf{p} - q\mathbf{A})^2
\end{align*}
Since $\mathbf{B} = \begin{bmatrix}0 & 0 & B\end{bmatrix}^T$, there are multiple choices of $\mathbf{A}$,
for example 
\begin{align*}
    \begin{bmatrix}
        0 \\ Bx \\ 0
    \end{bmatrix},~~
    \begin{bmatrix}
        -By \\ 0 \\ 0
    \end{bmatrix},~~
    \begin{bmatrix}
        -By/2 \\ Bx/2 \\ 0
    \end{bmatrix}
\end{align*}
I will adopt the last one to simply the calculation.
Suppose we have $A_x = -By/2$, $A_y = Bx/2$, $A_z=0$, then
\begin{align*}
    H &= \frac{1}{2m}[(p_x - qA_x)^2 + (p_y - qA_y)^2] \\
      &= \frac{1}{2m}(\alpha^2 2P_1\sin^2 Q_1 + \alpha^2 2P_1\cos^2 P_1)\\
      &= \frac{qB}{m}P_1
\end{align*}
In this case, the solution of EOM is
\begin{align*}
    Q_1 &= \frac{qB}{m} t + C_1\\
    Q_2 &= C_2\\
    P_1 &= C_3\\
    P_2 &= C_4
\end{align*}


\end{solution}




\begin{problem}
Use the method of infinitesimal canonical transformation to solve the motion of
a one-dimensional harmonic oscillator as a function of time.
\end{problem}

\begin{solution} 
Let $G=H$, $U=q$, then
\begin{align*}
    U' &= [q, H] = \frac{p}{m}\\
    U'' &= [p/m, H] = -\omega^2 q\\
    U''' &= -\omega^2 \frac{p}{m}\\
    U^{(4)} &= \omega^4 q
\end{align*}
Therefore
\begin{align*}
    q(t) &= q_0 + \frac{p_0}{m}t - \frac{1}{2!}\omega^2q_0 t^2 - \frac{1}{3!}
    \omega^3 \frac{p_0}{m} t^3 
    + \frac{1}{4!}\omega^4 q_0 t^4 + \cdots\\
         &= q_0\cos\omega t + \frac{p_0}{m\omega}\sin\omega t
\end{align*}
Follow the same way, we can also get $p(t)$
\begin{align*}
    p(t) &= -m\omega q_0\sin\omega t + p_0\cos\omega t
\end{align*}
where $q_0=q(0)$ and $p_0=p(0)$.

\end{solution}


% \end{multicols*}
\end{document}

